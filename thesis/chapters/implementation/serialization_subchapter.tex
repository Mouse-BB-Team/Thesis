\section{Data serialization}\label{sec:data-serialization}
The persisted data from the database is to serve as the input for a machine learning model, however, due to inconvenient usage of \gls{sql} queries for such purposes, the serialization tool was developed in order to translate the state of the database records to binary files.
In the presented solution, such files are treated as immutable, so the operation of serialization can be performed only once, which results in the improvement of the required time to read the data by the machine learning model.
Moreover, binary files allow straightforward sending and storing data in external infrastructures, such as PLGrid\footnote{\url{http://www.plgrid.pl/}}, because it does not require maintenance of database engine in that environment.
Such an approach also makes it possible to process data and prepare them in a way that is required by the machine learning model.

To fulfill these requirements, the serializer and deserializer tools were developed for saving sequences in binary files and further reading those data in Python script.
The serialization is performed using a tool written in Go\footnote{\url{https://golang.org/}} and provides additional tuning of resulted binary files and configuration of connection to the database.
Among the others, the tool allows choosing the type of event generated by the user such as mouse click or move, minimum sequence length that should be considered as valid data, minimum screen resolution in order to filter the actions from mobile devices or the time gap between two actions that should be considered as the boundary between two sequences.
The results of such filtration are saved in the chosen directory dividing the output into the user's directories and saving each separate sequence in a single file, so the output consists of many user's directories each containing many single sequence files.
It is also possible to use a so-called one-user mode that enables generating output data only for a single arbitrary chosen user for debugging purposes.

In order to make serialization uncomplicated and transferable between different programming languages, the serialization framework was used.
At the beginning the chosen one was Apache Avro\footnote{\url{https://avro.apache.org/}} which allows to defining the schema in simple \gls{json} file.
The serialization is performed with help of a library that allows reading schema and saving programming language native objects to binary files.
In the presented solution serialization should be performed using the library for Go and the deserialization with support of the library for Python, however, they proved to be incompatible which resulted in errors in deserialized data.

To avoid invalid data and to do not spend too much time on finding the bug in those libraries, another approach was taken by applying the Protocol Buffers\footnote{\url{https://developers.google.com/protocol-buffers}} technology.
Protocol Buffers or simply Protobuf is a method of serializing data to the binary form, but the real advantage of it is official multilingual support by generating a serializing code in required programming language.
Protobuf also requires the definition of a schema like Apache Avro, but unlike Avro, the config file format is developed especially for Protobuf, at the same time it is also readable and effortless to write.
Basing on the created schema the code was generated both for the serializer and the deserializer, but in the case of deserializer, the data is directly read to the Pandas\footnote{\url{https://pandas.pydata.org/}} Dataframe objects, which provides a simple interface to manipulate huge amount of data.
The deserializing tool was designed to work directly in the front of the neural network with additional preprocessing step, therefore it was written in Python to provide flexibility in adapting this feature in further work.
It is also able to read data from the directory tree created by the serializer which enables seamless integration between these two parts.