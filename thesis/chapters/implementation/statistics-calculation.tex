\subsection{Statistics calculation}\label{subsec:statistics-calculation}
The prepared model was intended to be running several times in order to measure the consistency and stability of the results.
To properly measure and visualize the model scores, a dedicated module for processing the results was prepared.
Two main core parts can be distinguished --- the plotting utilities and the statistics metrics calculation submodule.

The submodule for calculating a metrics allows for producing the following model statistics:
\begin{enumerate}
    \item Mean accuracy
    \item Mean loss
    \item Mean FAR\footnote{False Acceptance Rate}
    \item Mean FRR\footnote{False Rejection Rate}
    \item Mean true negatives
    \item Mean false negatives
    \item Mean true positives
    \item Mean false positives
\end{enumerate}

The submodule for plotting the charts allows for the following:
\begin{enumerate}
    \item Generic linear plot creation used to display accuracy and loss charts
    \item Creating the accuracy percentile histogram
\end{enumerate}

The above metrics and functionalities are designed to produce the averaged model results for multiple executions.
The developed model is stable and the error is irrelevant between each execution.
