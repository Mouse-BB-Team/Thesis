% How to ref to paragraph: https://tex.stackexchange.com/a/334756
\paragraph{Statistics calculation}\label{para:statistics-calculation}
The prepared model was intended to be running several times in order to measure the consistency and stability of the results.
To properly measure and visualize the model scores, a dedicated module for processing the results was prepared.
The two main core parts can be distinguished --- the plotting utilities and the statistics metrics calculation submodule.

The submodule for calculating metrics allows for producing the mean value of accuracy, model loss, percentage of \gls{far} and \gls{frr}, number of true negatives, true positives, false negatives and false positives.
The plotting submodule allows for producing the following charts:
\begin{samepage}
    \begin{itemize}
        \item[---] generic linear plot creation used to display accuracy and loss charts,
        \item[---] creating the accuracy percentile histogram.
    \end{itemize}
\end{samepage}

The above metrics and functionalities are designed to produce the averaged model results for multiple executions.
The developed model is stable and the error is irrelevant between each execution.
