\section{Machine learning model}\label{sec:machine-learning-model}
The machine learning model prepared and trained on the collected dataset was the part of the data evaluation final part.
The dataset collected during the thesis preparation is not sufficient for a proper and full machine learning process.
Google Cloud Services are able to handle the deployment of the whole system efficiently.
However, this form of hosting and maintaining the infrastructure of the application comes with its price.
The whole infrastructure was running for about two months and during this time generated the cost of a total \$300.
This amount of cash is the highest financial outlay that could have been incurred by the authors of the work since no other scholarship than PLGrid computation cluster was granted for the purpose of preparing the thesis.
Longer exposure on the web would cost extra money, that could not be afforded.
The other matter is that the engineering thesis defending has its deadline specified, thus the project has been carefully thought-out in the manner of time since the beginning of the implementation.
In order to meet the adopted milestones and goals, the data collection period had to follow strict deadlines.
The duration of the period when the data was collected could not be extended to broader terms.

The first remedy for a small amount of collected data was to use the transfer learning technique.
As stated in the "A survey of transfer learning" by Weiss et al. --- "In certain scenarios, obtaining training data that matches the feature space and predicted data distribution characteristics of the test data can be difficult and expensive.
Therefore, there is a need to create a high-performance learner for a target domain trained from a related source domain.
This is the motivation for transfer learning.
Transfer learning is used to improve a learner from one domain by transferring information from a related domain"~\cite{transfer-learning-def}.
The base feature vector for transferring to the bot recognition domain was taken from the TensorFlow Hub\upperref{itm:tensorflow-hub} website.
The model was pre-trained on ImageNet dataset\upperref{itm:image-net} using the architecture of Inception V3, which is showing great potential in the terms of the computer vision with improved performance over the previous versions, with a relatively modest computation cost~\cite{inception-v3}.
The model was built with the following dimensions of $299$\texttimes$299$\texttimes$3$ for the data input.
The topmost layers were added for the transferring and tuning to the bot detection domain.

The first attempts were not very exciting because of the limited amount of samples.
The model behaved unilaterally which means that the dominance of user's samples outshone the bot samples.
The different approaches were taken to improve the performance such as linear interpolation by connecting the points in recorded sequences.
Each used sequence originally consisted of many single discrete points without any additional pieces of information.
Interpolation provided an order between discrete coordinates and allowed feeding the neural network with additional information.

Another considered approach was a manipulation of the input data size.
The user's sequences were limited to the number of total bot sequences.
This solution was aimed to balance the dataset at the cost of fewer data.
The results of this approach turned out insufficient --- because of the total amount of bot sequences, the total size of the dataset drastically shrank, which resulted in a performance deterioration.
On the other hand, the duplication of the bot sequences was used.
The idea was similar to the one before, but instead of reducing, the number of bot samples was increased by using a single sample several times.
It resulted in an artificially balanced dataset.
However, this approach did not increase performance at all.

Manipulation of the distribution of labels between training and testing dataset was also considered.
It was done by performing either an equalization of the total number of both types in the testing dataset and the distribution of samples between both datasets.
The first solution did not affect model performance, but the second one slightly improved overall performance if the ratio was close to 50:50.
When the number of training samples was significantly greater than testing ones, the accuracy decreased due to a very small number of bot samples in testing dataset.

Yet another attempt to reduce the impact of the inappropriate dataset was changing the dataset itself.
The developed serializing tool made it possible to create a few datasets from recorded data with different minimal sequence length limits.
Using longer sequences meant that the overall number of them would be smaller.
The authors tested several ones and found out that the best performance was for a length equal to 50.

To overcome the described issue, another technique called data augmentation was used.
As stated in the "The Effectiveness of Data Augmentation in Image Classification using Deep Learning" by Perez et al. article --- "It is common knowledge that the more data an ML algorithm has access to, the more effective it can be.
Even when the data is of lower quality, algorithms can actually perform better, as long as useful data can be extracted by the model from the original data set"~\cite{augmentation}.
The main idea of this approach is to increase the samples of the dataset by using the same sample several times, but each time slightly transformed it.
The transformation is based on operations such as rotations, zooming, and flipping the original sample with some probability to do so, so each artificially generated sample should be slightly different than the others.
This approach allows to significantly increase the size of the dataset.
In the presented solution, each of the datasets (users and bot) was extended to 30,000 samples which results in a total of 60,000 samples with just 639.
The ratio between the training and testing part was set to 80:20, so the network used for the training 48,000 samples and for the testing 12,000.
For the optimization of the model, the optimizer called Adam was used along with the learning rate exponential decay set initially to \num{1e-4}.
The decay rate was equal to 0.96 and the decay steps were set to \num{1e5}.
The loss function was set to the binary cross entropy loss due to the binary classification problem.

To save time and optimize resources usage, the first approach for running the model was to execute it in a multi-threaded way.
For this purpose, the module which was facilitating that was created.
Soon this approach was abandoned due to the encountered problems.
The main problem was that the SLURM manager is not able to split the resources in a way to the multithreaded approach to be sensible.
The resources were assigned to the whole computing node on which only 2 \gls{gpu} cards were able to be allocated.
This caused the problem of many machine learning models trying to run on a single card.

When run on the Prometheus cluster, the better approach was to split the computation between two \gls{gpu}s and run the model sequentially.
For this purpose, the Mirrored Strategy distribution technique from the TensorFlow library was used whereas training is split across multiple replicas.
Thanks to the split and the computing power of \gls{gpu}, single execution is faster, thus can be multiplied several times.