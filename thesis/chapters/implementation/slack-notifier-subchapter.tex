\subsection{Slack notifier}\label{subsec:slack-notifier}
Working with PLGrid infrastructure and running a job on the Prometheus computing cluster was very helpful in terms of delegating the great amount of computing load onto external resources.
However, this solution has its downsides.
One of the major issues is the job queueing mechanism, which does not determine when the job will be executed and therefore it requires constant manual checking for results.
To avoid that somehow, the authors prepared a custom mechanism, that can monitor the status of the job and notify the Prometheus user through the Slack\footnote{\url{https://slack.com/} – a messaging app for teams} channel.
Three following message templates were prepared for the purpose of notification:
\begin{enumerate}
    \item Simple message --- Generic message type build as a \gls{json} template with help of Builder design pattern to have flexibility in customizing the message.
    \item Pending job message --- able to notify about: job title, reporter, commit hash, job registration date and time, header for app preview, info message taken from a commit.
    Build as a special case of Simple type message.
    \item Results message --- The most complex message with all the statistic results produced as a result of the model computation.
    Prepared as a \gls{json} template with the help of the Builder design pattern.
\end{enumerate}

To avoid that somehow, the authors prepared a custom mechanism, that can monitor the status of the job and notify the Prometheus user through the Slack channel using observability-like architecture.
