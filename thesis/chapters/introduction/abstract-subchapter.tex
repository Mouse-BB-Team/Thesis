This work consists of several chapters.
In the beginning, the authors describe the main concept of the conducted research alongside the related works that are an inspiration for this one.
The further chapters describe an implementation where the authors discuss the architecture of their solution along with problems during the development and obtained solutions.
The content of the mentioned chapters consists of several submodules that form a core of the presented solution.
In the beginning, a system for the data gathering is described (further called the Data Collection module), which was shared for the public usage during the research conduction.
The module architecture incorporates two applications --- backend API and frontend application --- that were deployed to the cloud and allowed to collect the data for the machine learning model.
The next module is the so-called Custom Bot module wherein the authors developed a bot that impersonates human-like users using the mathematical concept of Bézier curves.
The last submodule --- the Bot Detection is composed of the machine learning model itself and a set of tools that facilitate work with remote computational architecture where the model was trained and stored.
Among the mentioned tools, you can find serializer and deserializer tools that allow for effortless data preparation for usage in machine learning Python script as well as many statistics, preprocessing, and notification tools.

After the implementation description, the authors present the results of an adapted machine learning approach based on collected data during the research conduction.
The chapter also contains the limitations and conditions of the conducted research and the impact of them on the presented results.
The work ends with the conclusion along with the description of possible further study which in the authors' opinion may improve the obtained results.