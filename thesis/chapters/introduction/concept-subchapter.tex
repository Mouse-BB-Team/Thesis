\section{Abstract}\label{sec:abstract}
Cyber security is a hot topic, especially nowadays, when there are many threats and vulnerabilities in hardware as well as in software and the facilitated remote access to computers.
In the era of coronavirus, many people work from home which increases the risk of attacks because they are outside the internal company network and many of them use their own devices, thus could be easily hijacked.
People stores sensitive data on their phones and other devices exposing it to a potential risk of stealing, extortion, or blackmail.
Therefore security is necessary and nowadays it is developed and discussed dynamically in various applications and fields.
One of them is invisible user authentication that identifies the user based on his behavior such as keystroke dynamic or mouse dynamics.
The mentioned concept is relatively new because it requires sufficient computational resources and therefore could not be developed in the past.
The evolution of technology and machine learning approaches such as deep learning allows you to imagine behavioral biometrics as a possible future solution for securing applications and hardware.
The idea is fresh and therefore gives the potential to conduct the research, but it requires additional effort to collect the proper dataset for the machine learning model.
There exist some public datasets mentioned in related works such as the Balabit dataset\footnote{\url{https://github.com/balabit/Mouse-Dynamics-Challenge}} but they are not suitable for the problem of distinguishing a bot from a human.
Therefore the main ambition of this work is the creation of a system to gather such data from an example website alongside the creation of a bot that impersonates the human-like user.
The authors proposed also a machine learning model basing on collected data.

