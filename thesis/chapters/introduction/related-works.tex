\section{Related works}\label{sec:related-works}
The idea of using behavioral biometrics, especially based on mouse dynamics, is still a relatively new concept.
The more advanced papers on using the authentication systems that use a mouse movement were created in the late 2000s.
In 2007, Awad and Traore presented a work~\cite{firstMouseBBPapers1} in which they introduced a new form of behavioral biometrics based on mouse dynamics that can be used in different security applications.
The method is based on neural networks.
Later on, they continued their work and in 2009 presented paper~\cite{wang2009behavioral} that showed promising results.
However, in those works the ratios of \gls{far} (False Acceptance Rate) and \gls{frr} (False Rejection Rate) were reaching the values above the accepted commercial standards
Therefore, the systems were still not applicable for real-world scenarios.

With the recent technology improvements and standards, the detecting architectures were getting more and more accurate and robust, hence interesting to perform research on them.
The works like~\cite{a-deep-learning-approach-to-web-bot-detection-using-mouse-behavioral-biometrics} use deep learning techniques, such as convolutional neural networks, to achieve accuracy over the 96\%.
The authors gather the data with a custom site built for such a purpose.
However, they do not share details for reproducing this system.
In another work, Wang Kaixin et al.~\cite{a-user-authentication-and-identification-model-based-on-mouse-dynamics} use a support vector machine classifier to determine the identity of the user based on statistical and dynamical data taken from a low count of users.
The process of collecting data is unknown, as the lack of description also remains between different works related to the authentication based on the mouse dynamics.

The remedy for the lacking data gathering information would be to get access to publicly available, verified vast datasets of good quality.
One of the best available datasets is the Balabit dataset\upperref{itm:balabit}.
The mentioned dataset is used by Antal et al.~\cite{antal2019intrusion}, whose work shows the good quality of the data.
However, the given dataset has a couple of drawbacks, like short test sessions and overall lack of data abundance.
Chong et al.~\cite{Main} emphasize the lack of good data source as well and point at a Balabit dataset as the most adequate for the day.
The chosen 2D-\gls{cnn} (Convolutional Neural Network) approach shows interesting and promising results when evaluated on this dataset.
To come across the lack of datasets, we propose the architecture of the system for gathering the mouse data from users.
Furthermore, the user data along with the simulated bot actions are evaluated on the prepared model to show a potential and suitability for usage in a real machine learning model.
