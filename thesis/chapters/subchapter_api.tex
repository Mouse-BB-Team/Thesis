\section{Persistence and security}\label{sec:persistence-and-security}
The Data Collection module consists of several submodules, but the core of it is the backend API\footnote{Application Programming Interface} which was written in the 8\textsuperscript{th} version of Java\footnote{\url{https://www.java.com/}}.
The responsibilities of that API are mainly the persistence of captured data and user authentication.
The whole architecture of the API is based on the Spring\footnote{\url{https://spring.io/}} framework which is widely used for such purposes and provides many capabilities in fields of securing and persisting data.
Spring is divided into many projects that create compatible ecosystem and make it possible to develop scalable web applications.
In order to make development in Spring easier and faster the authors of the framework created Spring Boot\footnote{\url{https://spring.io/projects/spring-boot}} which sets up the environment for the developer and provides many easily imported libraries (so-called starters) to use.

To perform database operations using the idea of object–relational mapping (ORM) the implementation of Java Persistence API\footnote{\url{https://www.oracle.com/java/technologies/persistence-jsp.html}} (JPA) provided by Spring Boot was used.
The choice of the database was in the scope of relational databases because the captured data forms the time series which are easily mapped to the structure of the table.
The authors decided to use PostgreSQL\footnote{\url{https://www.postgresql.org/}} because of the popularity and wide range of support.
The data was stored in three different tables.
The first consists of the user's login, password and granted authority which was required to distinguish administrators from ordinary users.
The second one includes sequences of user's captured actions with timestamps, screen resolutions and types of the events.
The last one stores all possible action types with the description.

The collected data is sent to the API via Representational State Transfer (REST), which is nowadays one of the most popular interfaces between web applications, because of its simplicity and native support in many different programming languages and technologies.
The schema of the interface was established at the beginning of the work on this module, so it was possible to create the API and the fronted application simultaneously.

To protect the data from invalid properties, the input was also validated before saving to the database in such a way that all of persisted data contained all required elements.
Securing such an application from malware users is also important issue, so the access was restricted only for signed users that existed in the database.
The security was provided by using OAuth2\footnote{\url{https://oauth.net/2/}} and JSON Web Token\footnote{\url{https://jwt.io/}} (JWT).
The API was working as a OAuth2 provider which issued JWT tokens and authorized them basing on the user's credentials stored in the database.
Such a mechanism allowed to control access to the application and enforce registration before the use of the system.

The API was prepared with two different switchable profiles: admin oriented and user oriented.
They were created to provide flexibility in accounts creation~---~respectively~---~only by admin and anyone.
In further work only user oriented profile was used.