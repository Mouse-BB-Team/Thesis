\chapter{Summary}\label{ch:summary}

\section{Summary}\label{sec:summary}
During the work on the presented solution, the authors took steps to limit the impact of the imbalanced dataset.
As an example, linear interpolation was used.
It was done by connecting points in recorded sequences.
Each single recorded sequence consisted of many single discrete points.
Interpolation provided an order between discrete coordinates and gave the opportunity to feed the neural network with additional information.

Another approach that was taken was a manipulation of the input data size.
On the one hand, the user's sequences were limited to the number of total bot sequences.
This solution was aimed to balance the dataset at the cost of fewer data.
The results of this approach turned out insufficient.
Because of the total amount of bot sequences, the total size of the dataset drastically shrank, which resulted in a worsening of performance.
On the other hand, the duplication of the bot sequences was used.
The idea was similar to the one before, but instead of reducing, the authors increased the number of bot samples using a single sample several times.
It resulted in an artificially balanced dataset.
However, this approach did not increase performance at all.

Manipulation of the distribution of labels between training and testing dataset was also considered.
It was done by equalization of the number of both types of samples in testing dataset as well as manipulating the distribution of number of samples between both datasets.
The first solution had no effect, but the second one slightly improves overall performance if the ratio was close to \num{50}:\num{50}.
When the number of training samples was significantly greater than testing ones, the accuracy decreased due to a very small number of bot samples in testing dataset.

