\chapter{Summary}\label{ch:summary}

\section{Summary}\label{sec:summary}
During the work on the presented solution, the authors took steps to limit the impact of the imbalanced dataset.
As an example, linear interpolation was used.
It was done by connecting points in recorded sequences.
Each single recorded sequence consisted of many single discrete points.
Interpolation provided an order between discrete coordinates and allowed feeding the neural network with additional information.

Another approach that was taken was a manipulation of the input data size.
On the one hand, the user's sequences were limited to the number of total bot sequences.
This solution was aimed to balance the dataset at the cost of fewer data.
The results of this approach turned out insufficient.
Because of the total amount of bot sequences, the total size of the dataset drastically shrank, which resulted in a performance deterioration.
On the other hand, the duplication of the bot sequences was used.
The idea was similar to the one before, but instead of reducing, the authors increased the number of bot samples using a single sample several times.
It resulted in an artificially balanced dataset.
However, this approach did not increase performance at all.

Manipulation of the distribution of labels between training and testing dataset was also considered.
It was done by equalization of the number of both types of samples in testing dataset as well as manipulating the distribution of the number of samples between both datasets.
The first solution did not affect, but the second one slightly improved overall performance if the ratio was close to \num{50}:\num{50}.
When the number of training samples was significantly greater than testing ones, the accuracy decreased due to a very small number of bot samples in testing dataset.

Yet another attempt to reduce the impact of the inappropriate dataset was changing the dataset itself.
The developed serializing tool made it possible to create a few datasets from recorded data with different minimal sequence length limits.
Using longer sequences meant that the overall number of them would be smaller.
The authors tested several ones and found out that the best performance was for a length equal to \num{50}, as it was mentioned before.

All of the presented approaches tended to minimize the dataset problem.
Some of them slightly improved performance and those were considered in the final solution.
Despite the efforts of the authors, the described problem significantly worsened the performance of the model.

\section{Further study}\label{sec:further-study}
Due to the described issues, the authors find further study mainly in the improvement of the dataset.
Some efforts may be taken to extend the size of the recorded data.
Firstly, recording sequences of a bigger group of users may be proposed as a solution, especially bot users to prevent imbalance.
Such a solution should improve the overall performance of the model or at least suggest other problems related to the quality of the dataset.\\
If the quality of the recorded samples would be inappropriate, the collection module\footnote{\url{https://github.com/Mouse-BB-Team/Data-Collection}} should be reviewed and improved.
The main object of interest should be the method of gathering the samples.
Delays and synchronization that can disturb the reliability of the data may be considered as a major area of study.

The different approach that may be taken into consideration is another machine learning model.
In the presented solution, the focus was on the two-dimensional convolutional neural network, taking an example from related works like Chong et al.~\cite{Main} and Wei et al.~\cite{Inspiration}.
The work~\cite{Main} also shows other solutions, especially recursive neural networks.
Those kinds of neural networks are very popular in problems where the order and time intervals between samples have natural interpretations.
The problem which is described in this work also has similar properties.

These two described areas of study are found by the authors as the major to improve performance and reliability.
To deliver a safe and reliable solution to the commercial market further study is necessary.