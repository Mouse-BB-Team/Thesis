\section{Further study}\label{sec:further-study}
According to the described issues with the machine learning model, the authors find further study mainly in the improvement of the dataset.
Some efforts may be taken to extend the size of the recorded data.
Firstly, recording sequences of a bigger group of users may be proposed as a solution, especially enlarge the bot users to prevent imbalance.
Such a solution should improve the overall performance of the model or at least suggest other problems related to the quality of the dataset.

If the quality of the recorded samples should be improved, the collection module\upperref{itm:data-collection} ought to be reviewed.
The main object of interest should be the method of gathering the samples.
Delays and synchronization that can disturb the reliability of the data may be considered as a major area of study.

The different approach that may be taken into consideration is another machine learning model.
In the presented solution, the focus was on the two-dimensional convolutional neural network, taking an example from related works like Chong et al.~\cite{Main} and Wei et al.~\cite{a-deep-learning-approach-to-web-bot-detection-using-mouse-behavioral-biometrics}.
The work~\cite{Main} also shows other solutions, especially recursive neural networks.
Those kinds of neural networks are very popular in problems where the order and time intervals between samples have natural interpretations.
The problem which is described in this work also has similar properties.

These two described areas of study are found by the authors as the major to improve performance and reliability.
To deliver a safe and reliable solution to the commercial market further study is necessary.
