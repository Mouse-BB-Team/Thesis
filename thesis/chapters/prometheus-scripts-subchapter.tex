\section{Prometheus computing cluster}\label{sec:prometheus-computing-cluster}
Data processing and machine learning model training and evaluation require a lot of computing resources.
Performing such computations locally, on a single processing core is not the best approach, at least not the most effective method.

To enhance the process of manipulating the collected data, the computing cluster was used as a part of the educational grant issued by the PLGrid.
To make the most out of the computing cluster, plenty of helpers and steering scripts were prepared to automate the process of managing the Prometheus supercomputer.

Prometheus cluster is sharing the resources among multiple users.
Therefore, the computation is scheduled as a job by the SLURM Workload Manager\footnote{SLURM homepage --- \url{https://slurm.schedmd.com/documentation.html}}.
To submit a batch job to the SLURM manager, the input shell script is needed and in this case, the \textit{sbatch\_job\_config} script which contains the job description was prepared.
There is a declaration for requesting the 2 GPUs for the 5 hours of a total wall time with the main python script usage as a last step.

On the top of the \textit{sbatch} script, a helper shell script exists --- \textit{run-plgrid-job}, which is obtaining useful values that steer the SLURM scheduler as well as triggers the notification job described in previous subsection~\ref{subsec:slack-notifier}.

For automation, the Git version control system was used to implement a very simple continuous delivery system.
The combination of the git versioning system with its aliases allowed to create a \textit{deploy} alias, which was simply performing the pushing to the remote system --- in this case, the Prometheus cluster --- but also connecting to it and submitting the job.
After successfully submitting, the user receives a notification on Slack.
When the job starts, crashes, or finishes properly, the notification is sent again to the Slack.
This gives a very convenient insight into the process and progress of the job without constantly checking for the status manually.
The whole setup can be done easily via the \textit{make} program with prepared \textit{Makefile} file.
