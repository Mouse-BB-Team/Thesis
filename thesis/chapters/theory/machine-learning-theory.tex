\section{Machine learning approaches}\label{sec:machine-learning-approaches}
In modern cybersecurity, there exist many different threats that potentially expose the system to being compromised.
Many of them are very sophisticated, some are pretty difficult to distinguish from legitimate operations.
Nowadays, there is a trend to use machine learning approaches to meet the high demands for the quality and reliability of security systems.

As stated by Tony Thomas et al. in \cite{thomas2020machine} --- "Machine learning (ML) may be defined as the ability of machines to learn without being explicitly programmed.
Using mathematical techniques across cyberdata, ML algorithms can build models of behaviors and use those models as a basis for making predictions on newly input data".
This behavior of machine learning techniques is very convenient, especially when the domain of the problem and the borders between the data cannot be explicitly expressed by the written program.
Machine learning models can learn from their own mistakes and then recognize or even predict the future attacks.

In scope of sequential data with natural timespan interpretation such us mouse actions, the recursive neural networks are natural choice.
However, if the data can be represented as a bit map or picture, the convolutional neural networks become possible to use.
Due to the size of picture representation in a computer world, it is pretty expensive in the meaning of computational cost to use plain neural networks because the number of weights in such a network becomes tremendous.
Convolutional neural networks address this issue and make the computation significantly faster and efficient.

Keiron O'Shea et. al in \cite{cnn-description} define an architecture of \gls{cnn}'s as a connected network of three types layers: convolutional layer, pooling layer and fully-connected layer.
Thomas et. al in \cite{thomas2020machine} add to these layers so-called "rectified linear unit" layer.
The convolutional layer is combined out of filters that represent features of the images that the given filter should recognize.
These filters are then used on the different regions of images from the dataset by performing convolution.
The output reflects the found matches between the image that is recognized and the one described by the filter.
Pooling is a technique used to decrease the size of the image by grouping pixels and returning the representative value for this group.
As an example, the max-pooling bases on returning the pixel with the highest value in the group.
The fully-connected layer is a layer that builds plain artificial neural networks.
All of the neurons inside this layer are pairwise connected to the ones from the previous layer, but they are not connected inside the current layer.
The rectified linear unit (ReLU) layer mentioned in \cite{thomas2020machine} is in some sense responsible for adding a non-linearity to the neural network.
The authors of this work define its responsibility as "changing the negative pixel values in the image to zero, which gives us another stack of images with no negative values"~\cite{thomas2020machine}.
This layer is called an activation function because it activates the next layer only if the value of the pixel is positive.

The problem that this work raises can be specified as a binary problem because there exist only two possible categories for the data --- user's and bot actions.
In such situations, the measure of quality and correctness of the solution is commonly defined by a confusion matrix.
This matrix defines the performance of the model and consists of several measures: true positives, true negatives, false positives and false negatives.
True positives (TP) are defined as the number of samples that the model assign to the positive category and the assignment was correct.
The opposite to them are false positives (FP), which can be described as faulty categorized to the positive category.
The true negatives (TN) and false negatives are analogous, but the assigned category is negative.

Basing on described measures, one can define relative measures --- false rejection rate (FRR) and false acceptance rate (FAR).
They are defined as follows:

\[
    FRR = \frac{FN}{FN + TP} \hspace{8mm} FAR = \frac{FP}{FP + TN}
\]
The smaller their values, the better the performance of the model.

\textbf{Transfer learning + augmentation}
