\documentclass[en,11pt]{classes/aghdpl}  % praca w języku angielskim

% Lista wszystkich języków stanowiących języki pozycji bibliograficznych użytych w pracy.
% (Zgodnie z zasadami tworzenia bibliografii każda pozycja powinna zostać utworzona zgodnie z zasadami języka, w którym dana publikacja została napisana.)
\usepackage[english]{babel}

\usepackage[utf8]{inputenc}

% dodatkowe pakiety
\usepackage[dvipsnames]{xcolor}
\usepackage{listings}

\usepackage{mathtools}
\usepackage{amsfonts}
\usepackage{amsmath}
\usepackage{amsthm}
\usepackage{hyperref}
\usepackage[chapter,outputdir=build]{minted}

\usepackage[toc]{glossaries}

% --- < bibliografia > ---

\usepackage[
style=numeric,
sorting=nty,
%
% Zastosuj styl wpisu bibliograficznego właściwy językowi publikacji.
language=autobib,
autolang=other,
% Zapisuj datę dostępu do strony WWW w formacie RRRR-MM-DD.
urldate=iso8601,
% Nie dodawaj numerów stron, na których występuje cytowanie.
backref=false,
% Podawaj ISBN.
isbn=true,
% Nie podawaj URL-i, o ile nie jest to konieczne.
url=false,
%
% Ustawienia związane z polskimi normami dla bibliografii.
maxbibnames=3,
% Jeżeli używamy BibTeXa:
backend=bibtex,
]{biblatex}

\usepackage{csquotes}
% Ponieważ `csquotes` nie posiada polskiego stylu, można skorzystać z mocno zbliżonego stylu chorwackiego.
\DeclareQuoteAlias{croatian}{english}

\addbibresource{bibliography/bibliography.bib}

% Nie wyświetlaj wybranych pól.
%\AtEveryBibitem{\clearfield{note}}


% ------------------------
% --- < listingi > ---
%--- < json listing > ---
\definecolor{delim}{RGB}{20,105,176}
\definecolor{numb}{RGB}{106, 109, 32}
\definecolor{string}{rgb}{0.64,0.08,0.08}

\lstdefinelanguage{json}{
numbers=left,
numberstyle=\small,
frame=single,
rulecolor=\color{black},
showspaces=false,
showtabs=false,
breaklines=true,
captionpos=b,
postbreak=\raisebox{0ex}[0ex][0ex]{\ensuremath{\color{gray}\hookrightarrow\space}},
breakatwhitespace=true,
basicstyle=\ttfamily\small,
upquote=true,
morestring=[b]",
stringstyle=\color{string},
literate=
*{0}{{{\color{numb}0}}}{1}
{1}{{{\color{numb}1}}}{1}
{2}{{{\color{numb}2}}}{1}
{3}{{{\color{numb}3}}}{1}
{4}{{{\color{numb}4}}}{1}
{5}{{{\color{numb}5}}}{1}
{6}{{{\color{numb}6}}}{1}
{7}{{{\color{numb}7}}}{1}
{8}{{{\color{numb}8}}}{1}
{9}{{{\color{numb}9}}}{1}
{\{}{{{\color{delim}{\{}}}}{1}
{\}}{{{\color{delim}{\}}}}}{1}
{[}{{{\color{delim}{[}}}}{1}
{]}{{{\color{delim}{]}}}}{1},
}

% ---- YAML ----
\newcommand\YAMLcolonstyle{\color{red}\mdseries}
\newcommand\YAMLkeystyle{\color{black}\bfseries}
\newcommand\YAMLvaluestyle{\color{blue}\mdseries}

\makeatletter

% here is a macro expanding to the name of the language
% (handy if you decide to change it further down the road)
\newcommand\language@yaml{yaml}

\expandafter\expandafter\expandafter\lstdefinelanguage
\expandafter{\language@yaml}
{
keywords={true,false,null,n},
keywordstyle=\color{darkgray}\bfseries,
basicstyle=\YAMLkeystyle,                                 % assuming a key comes first
sensitive=false,
comment=[l]{\#},
captionpos=b,
numbers=left,
numberstyle=\small,
frame=single,
rulecolor=\color{black},
showspaces=false,
showtabs=false,
breaklines=true,
morecomment=[s]{/*}{*/},
commentstyle=\color{purple}\ttfamily,
stringstyle=\YAMLvaluestyle\ttfamily,
moredelim=[l][\color{orange}]{\&},
moredelim=[l][\color{magenta}]{*},
moredelim=**[il][\YAMLcolonstyle{:}\YAMLvaluestyle]{:},   % switch to value style at :
morestring=[b]',
morestring=[b]",
literate =    {---}{{\ProcessThreeDashes}}3
{>}{{\textcolor{red}\textgreater}}1
{|}{{\textcolor{red}\textbar}}1
{\ -\ }{{\mdseries\ -\ }}3,
}

% switch to key style at EOL
\lst@AddToHook{EveryLine}{\ifx\lst@language\language@yaml\YAMLkeystyle\fi}
\makeatother

\newcommand\ProcessThreeDashes{\llap{\color{cyan}\mdseries-{-}-}}

% ---- YAML end ----

% Użyj czcionki kroju Courier.
\usepackage{courier}

\usepackage{listings}
\lstloadlanguages{TeX}

\lstset{
	literate={ą}{{\k{a}}}1
           {ć}{{\'c}}1
           {ę}{{\k{e}}}1
           {ó}{{\'o}}1
           {ń}{{\'n}}1
           {ł}{{\l{}}}1
           {ś}{{\'s}}1
           {ź}{{\'z}}1
           {ż}{{\.z}}1
           {Ą}{{\k{A}}}1
           {Ć}{{\'C}}1
           {Ę}{{\k{E}}}1
           {Ó}{{\'O}}1
           {Ń}{{\'N}}1
           {Ł}{{\L{}}}1
           {Ś}{{\'S}}1
           {Ź}{{\'Z}}1
           {Ż}{{\.Z}}1,
	basicstyle=\footnotesize\ttfamily,
}

% ------------------------

\AtBeginDocument{
	\renewcommand{\tablename}{Tab.}
	\renewcommand{\figurename}{Fig.}
}

% ------------------------
% --- < tabele > ---

\usepackage{array}
\usepackage{tabularx}
\usepackage{multirow}
\usepackage{booktabs}
\usepackage{makecell}
\usepackage[flushleft]{threeparttable}

% defines the X column to use m (\parbox[c]) instead of p (`parbox[t]`)
\newcolumntype{C}[1]{>{\hsize=#1\hsize\centering\arraybackslash}X}


%---------------------------------------------------------------------------

\author{Kamil~Kaliś and Piotr~Kuglin}
\shortauthor{K.~Kaliś P.~Kuglin}

% \titlePL{Przygotowanie bardzo długiej i pasjonującej pracy dyplomowej w~systemie~\LaTeX}
% \titleEN{Protection of Web Applications with Behavioral Biometrics}

\titlePL{}
\titleEN{Protection of Web Applications with Behavioral Biometrics}


% \shorttitlePL{Przygotowanie pracy dyplomowej w~systemie \LaTeX} % skrócona wersja tytułu jeśli jest bardzo długi
\shorttitleEN{Protection of Web Applications with Behavioral Biometrics}

% \thesistype{Praca dyplomowa magisterska}
\thesistype{Bachelor's Thesis}

% \supervisor{prof. dr hab. Marcin Szpyrka}
\supervisor{Piotr Chołda, PhD}

% \degreeprogramme{Informatyka}
\degreeprogramme{Electronics and Telecommunications}

\date{2020}

% \department{Katedra Informatyki Stosowanej}
\department{Department of Telecommunications}

% \faculty{Wydział Elektrotechniki, Automatyki,\protect\\[-1mm] Informatyki i Inżynierii Biomedycznej}
\faculty{Faculty of Computer Science, Electronics and Telecommunications}

\acknowledgements{We hereby express our thanks to our supervisor – Piotr Chołda, PhD, who was guiding the process of creation of this thesis. The support in granting the needed resources for the study and feedback provided by Mr. Chołda are invaluable.\\
\newline
Special thanks to Krzysztof Rusek, PhD – for consulting the approach for proper machine learning model execution.\\
\newline
This research was supported in part by PLGrid Infrastructure.
}


\setlength{\cftsecnumwidth}{10mm}

%---------------------------------------------------------------------------
\renewcommand{\theparagraph}{}
\setcounter{secnumdepth}{4}
\brokenpenalty=10000\relax


\makeglossaries

\newglossaryentry{far}
{
name=FAR,
description={: False Acceptance Rate --- percentage of false recognition of users with granted access}
}

\newglossaryentry{frr}
{
name=FRR,
description={: False Rejection Rate --- percentage of false recognition of users that access was rejected}
}

\newglossaryentry{api}
{
name=API,
description={: Application Programming Interface}
}

\setglossarystyle{index}
\usepackage[toc]{appendix}

\begin{document}

\titlepages

% Ponowne zdefiniowanie stylu `plain`, aby usunąć numer strony z pierwszej strony spisu treści i poszczególnych rozdziałów.
\fancypagestyle{plain}
{
	% Usuń nagłówek i stopkę
	\fancyhf{}
	% Usuń linie.
	\renewcommand{\headrulewidth}{0pt}
	\renewcommand{\footrulewidth}{0pt}
}

\fancypagestyle{appendix}
{
\rhead{\textit{Appendix: Links}}
}

\fancypagestyle{bibliography}
{
\rhead{\textit{Bibliography}}
}

\newcounter{refcounter}
\newcommand{\upperref}[1]{\textsuperscript{\ref{#1}}}

\newcommand\blfootnote[1]{%
  \begingroup
  \renewcommand\thefootnote{}\footnote{#1}%
  \addtocounter{footnote}{-1}%
  \endgroup
}

\setcounter{tocdepth}{2}
\tableofcontents
\clearpage

\input{chapters/introduction/introduction-chapter}
    \section{Concept}\label{sec:concept}
Security is a hot topic, especially nowadays, when there are many threats and vulnerabilities in hardware as well as in software and the facilitated remote access to computers.
People stores sensitive data on their phones and other devices exposing it to a potential risk of stealing, extortion, or blackmail.
Therefore security is necessary and nowadays it is developed and discussed dynamically in various applications and fields.
One of them is invisible user authentication that identifies the user based on his behavior such as keystroke dynamic or mouse dynamics.
The mentioned concept is relatively new because it requires sufficient computational resources and therefore could not be developed in the past.
The evolution of technology and machine learning approaches such as deep learning allows you to imagine behavioral biometrics as a possible future solution for securing applications and hardware.
The idea is fresh and therefore gives the potential to conduct the research, but it requires additional effort to collect the proper dataset for the machine learning model.
There exist some public datasets mentioned in related works such as Balabit dataset\footnote{\url{https://github.com/balabit/Mouse-Dynamics-Challenge}} but they are not suitable for the problem of distinguishing bot from the human.
Therefore the main ambition of this work is the creation of a system to gather such data from an example website alongside the creation of a bot that impersonates the human-like user.
The authors proposed also a machine learning model basing on collected data.

    \section{Abstract}\label{sec:abstract}
This work consists of several chapters.
In the beginning, the authors describe the main concept of the conducted research alongside the related works that are an inspiration for this one.

The further chapters describe an implementation where the authors discuss the architecture of their solution along with problems during the development and obtained solutions.
The content of the mentioned chapters consists of several submodules that form a core of the presented solution.
In the beginning, a system for the data gathering is described (further called the Data Collection module), which was shared for the public usage during the research conduction.
The module architecture incorporates two applications --- backend API and frontend application --- that were deployed to the cloud and allowed to collect the data for the machine learning model.
The next module is the so-called Custom Bot module wherein the authors developed a bot that impersonates human-like users using the mathematical concept of Bézier curves.
The last submodule --- the Bot Detection is a set of several tools that facilitate work with remote computational architecture where the model was trained and also the machine learning model itself.
Among the mentioned tools, you can find serializer and deserializer tools that allow for effortless data preparation for usage in machine learning Python script as well as many statistics, preprocessing, and notification tools.

After the implementation description, the authors present the results of an adapted machine learning approach based on collected data during the research conduction.
The chapter also contains the limitations and conditions of the conducted research and the impact of them on the presented results.
The work ends with the conclusion along with the description of possible further study which in the authors' opinion may improve the obtained results.
    \section{Related works}\label{sec:related-works}
The idea of using behavioral biometrics, especially based on mouse dynamics, is still a relatively new concept.
The more advanced papers on using the authentication systems that use a mouse movement were created in the late 2000s.
In 2007, Awad and Traore presented a work~\cite{firstMouseBBPapers1} in which they introduced a new form of behavioral biometrics based on mouse dynamics that can be used in different security applications.
The method is based on neural networks.
Later on, they continued their work and in 2009 presented paper~\cite{wang2009behavioral} that showed promising results.
However, in those works the ratios of \gls{far} (False Acceptance Rate) and \gls{frr} (False Rejection Rate) were reaching the values above the accepted commercial standards
Therefore, the systems were still not applicable for real-world scenarios.

With the recent technology improvements and standards, the detecting architectures were getting more and more accurate and robust, hence interesting to perform research on them.
The works like~\cite{a-deep-learning-approach-to-web-bot-detection-using-mouse-behavioral-biometrics} use deep learning techniques, such as convolutional neural networks, to achieve accuracy over the 96\%.
The authors gather the data with a custom site built for such a purpose.
However, they do not share details for reproducing this system.
In another work, Wang Kaixin et al.~\cite{a-user-authentication-and-identification-model-based-on-mouse-dynamics} use a support vector machine classifier to determine the identity of the user based on statistical and dynamical data taken from a low count of users.
The process of collecting data is unknown, as the lack of description also remains between different works related to the authentication based on the mouse dynamics.

The remedy for the lacking data gathering information would be to get access to publicly available, verified vast datasets of good quality.
One of the best available datasets is the Balabit dataset\upperref{itm:balabit}.
The mentioned dataset is used by Antal et al.~\cite{antal2019intrusion}, whose work shows the good quality of the data.
However, the given dataset has a couple of drawbacks, like short test sessions and overall lack of data abundance.
Chong et al.~\cite{Main} emphasize the lack of good data source as well and point at a Balabit dataset as the most adequate for the day.
The chosen 2D-\gls{cnn} (Convolutional Neural Network) approach shows interesting and promising results when evaluated on this dataset.
To come across the lack of datasets, we propose the architecture of the system for gathering the mouse data from users.
Furthermore, the user data along with the simulated bot actions are evaluated on the prepared model to show a potential and suitability for usage in a real machine learning model.

\chapter{Theory}\label{ch:theory}

    \section{Behavioral Biometrics}\label{sec:behavioral-biometrics}
With the development of more and more powerful technology, there comes a need for sophisticated authentication methods.
One of a kind for authenticating a user with his individual physical patterns is static biometrics.
This approach is widely used nowadays in laptops or smartphones as a way for authentication.
The main concepts for this kind of biometrics are facial recognition, iris and fingerprint scanning.
However, these static methods raise big concerns.
First of all, the static approach is insecure, because when compromised, there is no way to change it dynamically.
Data are bounded in a way that cannot be deleted afterward.
Secondly, the big growth in technology, especially computer vision and cameras, raise a thread for facial recognition method, where the human face can be simply replicated based on a properly scanned subject head.
Moreover, this type of recognition raises concerns about human rights and privacy.
The IBM CEO, Arvinid Krishna in his letter~\cite{ibm_2020} to the US government deputies commit the withdrawal of facial recognition in the IBM technology stack.

The more sophisticated, private, and cost-effective implementation seems to be a behavioral biometric technique.
Behavioral biometrics are defined as "any quantifiable actions of a person. Such actions may not be unique to the person and may take a different amount of time to be exhibited by different individuals"~\cite{Yampolskiy2011}.
As stated by Tony Thomas --- these actions include voiceprints, signatures, typing patterns, keystroke patterns, or gait~\cite{thomas2020machine}.
The advantage of such an approach is the fact, that data collecting for further authentication can be gathered in a seamless manner, without the user's knowledge of the process.
Thanks to that, the properly designed system may be used to provide a continuous authorization of a user.
The malicious, hijacked session would be discovered because as stated by Yampolskiy --- "one of the defining characteristics of a behavioral biometric is the incorporation of the time dimension as a part of the behavioral signature"~\cite{Yampolskiy2011}, therefore the data is gathered and analyzed dynamically.
The domain can be divided into a couple of main categories, in which there exist the Human-Computer Interaction (HCI) based biometrics~\cite{Yampolskiy2011}.
In this approach, the data from different external devices is gathered along with its usage and then analyzed.
This device could be the keyboard with keystrokes analytics or mouse pointer usage.
This last seems to be the least discovered method due to time.
The mouse behavioral biometrics can find its usage in the web bot/web crawlers detection, as Wei et al.~\cite{a-deep-learning-approach-to-web-bot-detection-using-mouse-behavioral-biometrics} raise a topic of huge malicious bot traffic over the internet, which according to Bot Traffic Report 2016~\cite{bot-share-2016} reach to almost 30\% share of overall web bots traffic, thus the adoption of mouse behavioral biometrics would fit into the problem, hence "although web bots can
generate mouse events, it’s difficult for bots to perform mouse operations in a human manner"~\cite{a-deep-learning-approach-to-web-bot-detection-using-mouse-behavioral-biometrics}.
This works aims for broadening the knowledge in mouse behavioral biometrics.
    \section{Machine learning techniques}\label{sec:machine-learning-techniques}
In modern cyber security, there exist many different threats that potentially expose the system to being compromised.
Many of them are very sophisticated, some are pretty difficult to distinguish from legitimate operations.
Nowadays, there is a trend to use machine learning approaches to meet the high demands for the quality and reliability of security systems.

As stated by Tony Thomas et al. --- "Machine learning (\gls{ml}) may be defined as the ability of machines to learn without being explicitly programmed.
Using mathematical techniques across cyberdata, \gls{ml} algorithms can build models of behaviors and use those models as a basis for making predictions on newly input data"~\cite{thomas2020machine}.
This behavior of machine learning techniques is very convenient, especially when the domain of the problem and the borders between the data cannot be explicitly expressed by the written program.
Machine learning models can learn from their own mistakes and then recognize or even predict the future attacks~\cite{thomas2020machine}.

In scope of sequential data with natural timespan interpretation such us mouse actions, the recursive neural networks are natural choice.
As stated by Chong et al. --- "Given the sequential nature of mouse movement data, a recurrent neural network (\gls{rnn}), commonly employed for time series structure data, would be an intuitive choice for tackling this problem"~\cite{Main}.
However, if the data can be represented as a bit map or a picture, the convolutional neural networks become possible to use.
Due to the size of picture representation in a computer world, it is pretty expensive in the meaning of computational cost to use plain artificial neural networks because the number of weights in such a network becomes tremendous.
Convolutional neural networks address this issue and make the computation significantly faster and efficient.

Keiron O'Shea et. al in \cite{cnn-description} define an architecture of \gls{cnn}'s as a connected network of three types layers: convolutional layer, pooling layer and fully-connected layer.
The convolutional layer is combined out of filters that represent features of the images that the given filter should recognize.
These filters are then used on the different regions of images from the dataset by performing convolution.
The output reflects the found matches between the image that is recognized and the one described by the filter.
Pooling is a technique used to decrease the size of the image by grouping pixels and returning the representative value for this group.
As an example, the max-pooling bases on returning the pixel with the highest value in the group.
The fully-connected layer is a layer that builds plain artificial neural networks and it is built out of the neurons that are connected to the neurones from the previous layer, but they are not connected inside the current layer.
Thomas et. al add to these layers so-called rectified linear unit layer (\gls{relu}) and define its responsibility as "changing the negative pixel values in the image to zero, which gives us another stack of images with no negative values"~\cite{thomas2020machine}.
This layer is called an activation function because it activates the next layer only if the value of the pixel is positive.

The problem that this work raises can be specified as a binary problem because there exist only two possible categories for the data --- user's and bot actions.
In such situations, the measure of quality and correctness of the solution is commonly defined by a confusion matrix.
This matrix defines the performance of the model and consists of several measures: true positives, true negatives, false positives and false negatives.
True positives (\gls{tp}) are defined as the number of samples that the model assign to the positive category and the assignment is correct.
The opposite to them are false positives (\gls{fp}), which can be described as faulty categorized to the positive category.
The true negatives (\gls{tn}) and false negatives (\gls{fn}) are analogous, but the assigned category is negative.

Basing on the described measures, one can define relative measures --- false rejection rate (\gls{frr}) and false acceptance rate (\gls{far}).
These are defined as follows:

\begin{samepage}
\begin{equation}
    FRR = \frac{FN}{FN + TP}\label{eq:frr}
\end{equation}
\begin{equation}
    FAR = \frac{FP}{FP + TN}\label{eq:far}
\end{equation}
\end{samepage}

The values of \gls{frr} and \gls{far} are often expressed as a percentage value and the smaller their values, the better the performance of the model.
In authorization related problems such as raised in this work, there is a desire to have those values as low as possible because low \gls{far} defines the good security level of the system --- the lower it is, the less unauthorized users have access to the system, and accordingly \gls{frr} has an impact on authorization rejections of legitimate users.

In the presented solution, the authors use a transfer learning technique.
As stated in the "A survey of transfer learning" by Weiss et al. --- "In certain scenarios, obtaining training data that matches the feature space and predicted data distribution characteristics of the test data can be difficult and expensive.
Therefore, there is a need to create a high-performance learner for a target domain trained from a related source domain.
This is the motivation for transfer learning.
Transfer learning is used to improve a learner from one domain by transferring information from a related domain"~\cite{transfer-learning-def}.
This approach provides several conveniences, such as computational time and cost reduction.
The pretrained network does not require such a long time as in the case of training from scratch which also translates into the cost of calculations and in the end does not require a large computational grant.
The main advantage of such an approach is the reduction of the required amount of representative data.
In cases where the data gathering is difficult or the collected dataset is small, transfer learning seems to be a technique that is worth considering.
As a transfer learning model, the authors chose the InceptionV3\upperref{itm:inceptionV3} architecture from Tensorflow Hub\upperref{itm:tensorflow-hub}, which originally was trained on ImageNet\upperref{itm:image-net} dataset.
Basing on several benchmarks, it was considered that the described architecture fits well to the raised problem and the performance is great at the same time.
The architecture is build out of convolutional layers, average and max pooling, dropouts, concatenation layers and fully-connected layers.
Description of the InceptionV3 architecture is out of the scope of this work.

In order to perform the learning process of the model, the main dataset should be divided into two subsets --- training and test datasets.
The network basing on the output of the loss function is able to improve its predictions and make progress in classification.
In the case of the supervised learning, where the network learns from the given examples, there is also a requirement for the assignment of the labels that are considered as descriptive ones for the input sample.
When the gathered data consist of several categories, among which there is a dominant class in the meaning of volume of samples, such dataset is considered as an imbalanced one.
This issue has a negative impact on the learning process and should be resolved in order to improve the performance of the model.

\chapter{Implementation}\label{ch:implementation}
To fulfill assumptions of this work, the authors prepare a whole system which is able to collect and verify the mouse dynamics data.
The high-level flow of the system is presented in Fig.~\ref{fig:overall_system_structure} and provides the basis for further considerations.
\begin{figure}[!hbt]
    \includegraphics[width=\linewidth]{resources/overall_system_structure}
    \captionof{figure}{Overall system structure diagram}
    \label{fig:overall_system_structure}
\end{figure}

The entry point for the system is user's browser that allows the human user as well as human impersonating bot for access to the prepared website which is served in the Internet.
The \mbox{Data Collection}\upperref{itm:data-collection} module which is persisted and operates within a cloud acts as a mouse dynamics data collector, which means that every single mouse event generated by user on the website is intercepted, transformed and stored in the underlying database (Fig.~\ref{fig:overall_system_structure}, pt. 1 and pt.2).
The administrator of presented system is able to retrieve the data from the database in any time (pt. 3).
The downloaded data can be further processed (pt. 4) to the dataset which will be used in machine learning stage.
The administrator of the system should prepare a machine learning model and upload it to the Git repository (pt. 5) which will be then obtained by computing cluster to perform computation using the model from the observed branch.
Such an approach allows to work on the solution simultaneously by many data scientists and makes it possible to accelerate the research.
The dataset should be uploaded to the computing cluster and persisted in the group's storage that allows to use it by many different paralleled computations (not included in the diagram due to decrease of readability).
Each computation is requested from the local computer using prepared Git's alias that performs sequence of operations such us establishing the connection to the cluster, fetching the current version of code from the repository and submitting the job to the Slurm\upperref{itm:slurm} workload manager (pt. 6, 7, 8).
The work of the cluster is fully asynchronous because each job is queued and therefore the completion time is unknown which is very inconvenient from the authors' point of view because according to the authors' knowledge there not exists any notification system which allows getting the information about the finished job.
To overcome these limitations, the notification system is proposed as a part of Bot Detection\upperref{itm:bot-detection} module.
The creation of such system enables the possibility for presenting the results of the computed job in the notification message as well as the graphs prepared based on the output of the job.
To provide the possibility of sending the images, the notification system uses the external image hosting website called Imgur\upperref{itm:imgur} which allows to upload pictures to the server and host them under the generated \gls{url} (pt. 11).
The results and the graphs are combined into Slack's\upperref{itm:slack} message and then sent to the previously prepared Slack's channel by using the Slack's webhook \gls{api} (pt. 12, 13).

The further subchapters treats about the implementation details, at the beginning the Data Collection module is described along with the cloud configuration and performance tests, further, the bot which impersonates the human user and finally the machine learning model alongside the tools such as, among others, a notification module.


    \section{Persistence and security}\label{sec:persistence-and-security}
The Data Collection module consists of several submodules, but the core of it is the backend API\footnote{Application Programming Interface} which was written in the 8\textsuperscript{th} version of Java.
The responsibilities of that API are mainly the persistence of captured data and user authentication.
The whole architecture of the API is based on the Spring\footnote{\url{https://spring.io/}} framework which is widely used for such purposes and provides many capabilities in fields of securing and persisting data.

Spring is divided into many projects that create compatible ecosystem and make it possible to develop scalable web applications.
In order to make development in Spring easier and faster the authors of the framework created Spring Boot\footnote{\url{https://spring.io/projects/spring-boot}} which sets up the environment for the developer and provides many easily imported libraries (so-called starters) to use.

To perform database operations using the idea of object–relational mapping (ORM) the implementation of Java Persistence API\footnote{\url{https://www.oracle.com/java/technologies/persistence-jsp.html}} (JPA) provided by Spring Boot was used.
The choice of the database was in the scope of relational databases because the captured data forms the time series which are easily mapped to the structure of the table.
The authors decided to use PostgreSQL because of the popularity and wide range of support.
The data was stored in three different tables.
The first consists of the user's login, password and granted authority which was required to distinguish administrators from ordinary users.
The second one includes sequences of user's captured actions with timestamps, screen resolutions and types of the events.
The last one stores all possible action types with the description.

The collected data is sent to the \gls{api} via Representational State Transfer (REST), which is nowadays one of the most popular interfaces between web applications, because of its simplicity and native support in many different programming languages and technologies.
The schema of the interface was established at the beginning of the work on this module, so it was possible to create the \gls{api} and the fronted application simultaneously.

To protect the data from invalid properties, the input was also validated before saving to the database in such a way that all of persisted data contained all required elements.
Securing such an application from malware users is also important issue, so the access was restricted only for signed users that existed in the database.
The security was provided by using OAuth2\footnote{\url{https://oauth.net/2/}} and JSON Web Token\footnote{\url{https://jwt.io/}} (JWT).
The API was working as a OAuth2 provider which issued JWT tokens and authorized them basing on the user's credentials stored in the database.
Such a mechanism allowed to control access to the application and enforce registration before the use of the system.

The API was prepared with two different switchable profiles: admin oriented and user oriented.
They were created to provide flexibility in accounts creation~---~respectively~---~only by admin and anyone.
In further work only user oriented profile was used.
    \section{Presentation and reverse proxy}\label{sec:reverse-proxy}
The second core submodule created for the purpose of collecting the data from the users is the frontend application exposed the internet. This can be split further into two different functional modules~--- client and server side.

\subsection{Client side module}\label{subsec:client-side-module}
The client-side module is an end-user presentation layer that is built with the HTML5\footnote{\url{https://html5.org/}}~and~CSS3\footnote{\url{https://www.w3.org/Style/CSS/Overview.en.html}}~--- technologies that are the core and standard for modern web applications nowadays.
HTML is a widely used markup language used to create hypertext documents and CSS is used as a style-sheet for adding the design for web documents.
The content of the website is a free, prebuilt template taken from the Colorlib\footnote{\url{https://colorlib.com/}} collection.
Offered web templates are licensed under the CC BY 3.0 License \footnote{Creative Commons Attribution 3.0 License --- \url{https://creativecommons.org/licenses/by/3.0/}}.
The mechanism that collects users' mouse actions is designed as a "plugin" script, meaning that the template can be changed easily, so different styled environments that serve various purposes can be used to collect the data.

To use the system and participate in the research, interested user is required to first accept the consents of usage of the website as well as accept the usage of the cookies.
After that, the register option is allowed and required.
Next, the user is redirected to the login page.
To persists consistency of the registration and login, the custom static web pages were created to be easily transferable between the templates.
On each document, the FAQ panel with more information regarding the project is exposed as a drop-down list.
To authenticate, users are required to provide the credentials to log in to the system.
These credentials are then sent through https --- which means that they are secured and encrypted --- to the reverse proxy server which then performs some action to authenticate a user, sets the cookie with granted JWT\footnote{\url{https://jwt.io/}} token and redirects the user to the homepage.
More on the user authentication and the token obtaining sequence is described in
The whole sequence is described and shown in subsection \ref{subsec:server-side-module}).

After a successful login, the token allows for site usage and ensures the identity of the user.
From now on, the actions performed by the user are recorded into the batches and sent to the API every two seconds.
The event listeners are awaiting four different action types: mouse move, mouse-down, mouse-up, mouse wheel action.
Collected actions are packed into JSON object and sent to the server side API (described in \ref{subsec:server-side-module}) --- the fields included are shown in the Listing~\ref{listing:mouse-events-json-schema}.

\lstinputlisting[language=json, caption=JSON schema for mouse event batches, label=listing:mouse-events-json-schema]{resources/mouse-event-json-schema.json}

\subsection{Server side module}\label{subsec:server-side-module}
The server side module is build with Node.js\footnote{\url{https://nodejs.org/}} runtime with the Express\footnote{\url{https://expressjs.com/}} framework on the top of it.
Node.js is an asynchronous JavaScript runtime, which is widely used to build high-end, scalable commercial applications.
Express is light-weight and simple to use, yet powerful web framework for Node.js, which allows for a fast and convenient HTTP server set-up server for serving the static web documents over the internet.
This module is serving the purpose of reverse proxy between clients and the backend API.

The main responsibilities of the reverse proxy are signing up the user, token granting, validation and caching, serving static web documents storing and passing the data to for persistence.

When a new user tries to sign-up, secured with TLS credentials from the sign-up form are sent to one of the proxy endpoints, where they are then passed to the backend API.
Because the backend is not exposed anywhere over the public network, there is no need to secure the credentials with TLS.\\
The user is granted JWT Access and Refresh tokens after logging in using the login form.
The user credentials are secured with TLS and received by proxy API, whereas the proxy server is additionally appending the Client id and Client secret for OAuth2 server to the request "Authorization" header, as the proxy server authenticates to OAuth2 server with HTTP Basic authentication scheme.
Credentials of the user that wishes to log in are included in the body of the request.
When the user is properly authenticated, the OAuth2 server responds with a valid JWT token which is then set as a cookie with HttpOnly, Secure and SameSite:~strict options.
When the token is successfully granted, it is also saved to the Redis, which is a very efficient key-value NoSQL database.
The retrieval of such a token is very fast, so this is serving the purpose of the caching system, which results in a great efficiency improvement and lower response time of the server.
For each interaction and request for resources, the user has to hold a valid JWT Token.
First, the token is searched in the cache database. If it does not figure there, the proxy server is attempting to check the token with the backend OAuth2 server.
If the access token is not set on the user request or the server responds with Bad Request status, the refresh token is being used for access token renewal.
The complete seqnece visualization can be seen in \mbox{Fig.~\ref{fig:jwt-sequence}}.
The proxy server is storing the user mouse data received from the client-side and periodically passes it to the backend API for persistence.




% Propably should be included elsewhere, maybe appendix?
\begin{figure}[!hbt]
    
    \centering
    \includegraphics[width=\linewidth]{resources/jwt_sequence_diagram.png}
    \captionsetup{width=\linewidth}
    \captionof{figure}{JWT token obtaining sequence}
    \label{fig:jwt-sequence}
\end{figure}
    \section{Performance tests}\label{sec:performance-tests}
Solutions presented in previous chapters were deployed to the Heroku Cloud Application Platform\footnote{\url{https://www.heroku.com/}} to deliver the possibility of using the system among a wide range of users.
Heroku as a cloud provider has a free pricing plan for non-commercial apps, which is suitable for scientific usage such as this work.
The deployment process is performed using the idea of containerization based on Docker\footnote{\url{https://www.docker.com/}} which is a pretty convenient way of application delivery to the web server.
To enable such a possibility, the \mbox{Dockerfiles} with container configuration description were prepared for both frontend and backend services and also docker-compose configuration files were created to allow easy testing and development in the local environment.

In order to deliver reliable and efficient product to the users, the performance tests were prepared using JavaScript programming language in Node.js environment.
Those tests cover mainly the token exchange and refreshment and also allow to test the behavior of application during increased load produced by using the system by many different users.
The results of the tests showed that the Heroku free pricing plan had some limitations regarding database storage volume and response time from the application.
It turned out that the database in such a plan permit to store only 20,000 rows overall which is unacceptable for the data stream with the time resolution measured in milliseconds because a single user may cause exceeding the limit in a relatively short time.
Moreover, the described cloud system suspends the working of the application when it is idle and the waking up time is very long which is manifested in a usage lag after a short period.
Described issues with the Heroku Cloud Application Platform forced the authors to find another suitable cloud provider.
    \section{Deployment and orchestration}\label{sec:deployment-and-orchestration}
In order to overcome the issues described in the previous chapter, the cloud providers' research was conducted, and based on the results it was decided to adopt the Google Cloud Platform\footnote{\url{https://cloud.google.com/}} (GCP) as a deployment environment.

Unlike Heroku, GCP had no free pricing plan, but it did not restrict the usage of resources and therefore it met the project assumptions, and moreover, it had some extra advantages like native support for containers orchestration.
The latter was provided by using Kubernetes\footnote{\url{https://kubernetes.io/}}, the open-source solution introducing flexible deployments, scaling and container management.
Those features drastically simplify working with application development in a cloud but require additional setup and configuration files.

Kubernetes natively supports the usage of Docker containers so the ones prepared before could be reused, but the deployment process requires a so-called Deployment configuration files that enables tuning the resources assigned to the single instance of an application and also the number of mirror deployments of the application.
The basic unit managed by Kubernetes is Pod which may consist of many single Docker containers, but in this work the Pod is associated with a single container of application.
The described scaling method is known as Horizontal Pod Scaling, and it improves reliability and allows to increase the limit of the maximal load accepted by a single deployment of the application because the load is split into the mirror instances of the same application that work in parallel.

To deliver the load to all of the mirror applications, Kubernetes uses Services as an entry point to a group of Pods that are managed by one Deployment.
The Service exposes the Deployment under the single DNS\footnote{Domain Name System} name and updates the underlying IP address in cases of its change, but it also works as a load balancer which distributes the load among the managed Pods.
Kubernetes has several different Service types but in this work two of them were used: Cluster IP that exposes the Deployments in the scope of Kubernetes but hides them from external access and NodePort that permits the access using port-forwarding.
The latter one was dictated by the usage of Google Cloud HTTP(S) load balancer and it also required additional configuration by using an Ingress and obtaining the static external IP address.

To provide encryption of connection, Transport Layer Security protocol (TLS) should have been used, but requires the SSL Server Certificate, which can be issued only for existing valid domain names, so the authors were forced to buy such a domain name, while the certificate was issued by GCP.

The configuration also included some sensitive data, such as database credentials, OAuth2 secret, and SSH-RSA keys, thus the Secrets and ConfigMaps were used.
In contrast to the Redis cache database, the main database instance was not deployed in Kubernetes itself, but the Cloud SQL service with \mbox{PostgreSQL} was used.

    \section{Custom bot}\label{sec:custom-bot}
In order to prepare a dataset that consists of two classes --- the valid user and human impersonating bot, the custom bot was prepared.

The main idea behind this module is to first create scenarios that next could be executed by the bot software.
To achieve the most convenient creation of scenarios, the authors prepared a utility to record the coordinates and actions of the mouse.
Using this approach, the bot could impersonate a user in a more natural way.
Recorded mouse events are stored as a YAML\footnote{\url{https://yaml.org/}} schema, which is a human-readable standard for data serialization.
The structure of a YAML scenario schema is shown in Listing~\ref{listing:bot-events-yaml-schema}.

\lstinputlisting[language=yaml, caption=YAML schema for bot event scenarios, label=listing:bot-events-yaml-schema]{resources/bot-scenario-schema.yaml}

After the scenarios are recorded, the mechanism to execute them is required.
The Custom Bot executor script reads the YAML file and loads it to the memory as a list of events, that is then traversed.
The executor script performs an action stored as an "event" filed at the coordinates $(x, y)$.
If the coordinates are different than the current mouse position, the cursor is moved, but to make it more natural, movement is based on the Bézier curve concept.
The Bézier curves are used to effectively represent a smooth curve on a computer screen, as stated by B. T. Bertka~\cite{bezier-curves}. Knowing this, this concept can be used to simulate smooth human-like movements.

To be able to run a set of scenarios, the module introduces a so-called "batch runner" for predefined cases.
This automation saves a lot of time and effort because a set of scenarios can be executed at once to generate data for the bot class.


    \section{Data serialization}\label{sec:data-serialization}
The persisted data from the database is to serve as the input for a machine learning model.
However, due to inconvenient usage of \gls{sql} queries for such purposes, the serialization tool was developed in order to translate the state of the database records to binary files.
In the presented solution, such files are treated as immutable, so the operation of serialization can be performed only once, which results in the improvement of the required time to read the data by the machine learning model.
Moreover, binary files allow straightforward sending and storing data in external infrastructures, such as PLGrid\upperref{itm:plgrid}, because it does not require maintenance of database engine in that environment.
Such an approach also makes it possible to process data and prepare them in a way that is required by the machine learning model.

To fulfill these requirements, the serializer and deserializer tools were developed for saving sequences in binary files and further reading those data in Python script.
The serialization is performed using a tool written in Go\upperref{itm:golang} and provides additional tuning of resulted binary files and configuration of connection to the database.
Among the others, the tool allows choosing the type of an event generated by the user, such as mouse click or move, minimum sequence length that should be considered as valid data, minimum screen resolution in order to filter the actions from mobile devices or the time gap between two actions that should be considered as the boundary between two sequences.
The results of such filtration are saved in the chosen directory dividing the output into the user's directories and saving each separate sequence in a single file, so the output consists of many user's directories each containing many single sequence files.
It is also possible to use a so-called one-user mode that enables generating output data only for a single arbitrary chosen user for debugging purposes.

In order to make serialization easy and transferable between different programming languages, the serialization framework was used.
At the beginning, Apache Avro\upperref{itm:avro} was chosen.
It allows for defining the schema in a simple \gls{json} file and the serialization is performed with help of the library that allows reading schema and saving programming language native objects to binary files.
In the presented solution, the serialization should be performed using the library for Go and the deserialization with support of the library for Python.
Unfortunately, they proved to be incompatible which resulted in errors in deserialized data.

To avoid invalid data and to do not spend too much time on finding the bug in those libraries, another approach was taken by applying the Protocol Buffers\upperref{itm:protobuf} technology.
Protocol Buffers, or simply Protobuf, is a method of serializing data to the binary form, but the real advantage of it is official multilingual support by generating a serializing code in a required programming language.
Protobuf also requires the definition of a schema like Apache Avro, but unlike Avro, the config file format is developed especially for Protobuf.
At the same time it is also readable and effortless to write.
Basing on the created schema, the code was generated both for the serializer and the deserializer, but in the case of deserializer, the data is directly read to the Pandas\upperref{itm:pandas} Dataframe objects, which provides a simple interface to manipulate huge amount of data.
The deserializing tool was designed to work directly in the front of the neural network to reconstruct the serialized binary data.
Therefore it was written in Python to provide flexibility in adapting this feature in further work.
It is also able to read data from the directory tree created by the serializer which enables a seamless integration between the data gathering system and Bot Detection module\upperref{itm:bot-detection}.
    \section{Prometheus computing cluster}\label{sec:prometheus-computing-cluster}
Data processing and machine learning model training and evaluation require a lot of computing resources.
Performing such computations locally, on a single processing core, is not the best approach, at least not the most effective method.
To enhance the process of manipulating the collected data, the computing cluster was used as a part of the educational grant issued by the PLGrid.
To take the full advantage of the computing cluster, plenty of helpers and steering scripts were prepared to automate the process of managing the Prometheus supercomputer.
Prometheus cluster is sharing the resources among multiple users.
Therefore, the computation is scheduled as a job by the SLURM Workload Manager\upperref{itm:slurm}.
To submit a batch job to the SLURM manager, the input shell script is needed and in this case, the \textit{sbatch\_job\_config} script which contains the job description was prepared.
This includes all necessary configuration parameters for the SLURM manager, like job name, time, grant name, nodes count, GPU partition selection, CPUs allocated with the memory, modules added to properly run the main Python script.
On the top of the \textit{sbatch} script, a helper shell script exists --- \textit{run-plgrid-job}, which is obtaining useful values that steer the SLURM scheduler as well as triggers the notification job described in the paragraph devoted to~\nameref{para:slack-notifier} in the Section~\ref{sec:bot-detection-module}.
These notifications are really convenient, because one don't need to constantly check for the job status, but instead for every state, there is an appropriate notification for job schedule, running, crashed, and finished job.


For automation, the Git version control system was utilized to implement a very simple continuous delivery system.
For delivering and scheduling the job on remote host, the \textit{git deploy} alias was created and can be set up using the \textit{create-git-alias} script.
When alias is created and used, the script named \textit{git-push-alias} is launched which triggers the \textit{git push} command to update the repository and next the routine for executing the job on remote host is triggered with \textit{run-commit-execution} script.
After successfully submitting, the user receives a notification on Slack.
When the job starts, crashes, or finishes properly, the notification is sent again to the Slack.
This gives a very convenient insight into the process and progress of the job without constantly checking for the status manually.
The whole setup can be done easily via the \textit{make} program with prepared \textit{Makefile} file.
The routine sets all necessary information on local and remote system.
User needs to provide the PLGrid username, branch name from repository to observe and execute the model from on the Prometheus cluster.
Routine is also generating the SSH RSA key-pair for logging in seamlessly.
At the end, the necessary environmental variables are exported.

    \section{Data preprocessing}\label{sec:data-preprocessing}
To fulfill the requirements of the adapted transfer learning model the preprocessing tool was developed alongside a whole bunch of settings.
The collected data sequences are in form of coordinates' series, so it is necessary to represent them as 3-dimensional pictures which are the input to the convolutional neural network.
The described tool allows transferring those sequences into multidimensional arrays of integers and also scale them to the required input size of a used transfer learning model.

Moreover, to improve performance efficiency the isolated points on the "painting" are interpolated using a linear interpolation mechanism.
It was verified on the collected dataset that interpolated data results in better accuracy of a model in comparison to the isolated ones.
The presented tool works also as a splitter between training and testing sets using the provided ratio between them and as a label assigner which allows adding the corresponding labels to the single sequence basing on the provided identifier of the bot user.

Using this tool it is also possible to increase the number of bot samples by using several repetitions of the original bot set in cases of an imbalanced dataset.
The output of the preprocessor is a tuple of training and testing dataset along with labels, which can be directly the input of a neural network model.
    \section{Bot Detection module}\label{sec:bot-detection-module}
To validate the consistency and quality of the gathered data, the Bot Detection module\upperref{itm:bot-detection} is introduced.
This module includes various tools and utilities helpful for manipulating the data, model observations and results analysis.
The selected submodules are briefly described in the following paragraphs.
    \subsection{CSV writer}\label{subsec:csv-writer}
To persist machine learning model execution results and statistics for later analysis, the outcome data should be stored in a consistent way.
The CSV\footnote{Comma-Separated Values} file is a convenient file format to persist this kind of data because the values from different execution sessions can be easily appended using established schema.
The CSVWriter class introduces methods for appending the data to the file. The other helper methods allow for checking the correctness of the file and its content.

    \subsection{Uploading charts on web}\label{subsec:charts-web-uploading}
To see whether the machine learning model is actually getting better during the training process, accuracy and loss charts can be used as a really helpful method to determine the learning curve.
To receive a "live" preview of the model execution results with charts as a notification in Slack communicator, the charts needed to be stored somewhere on the internet.
The easiest way to achieve this requirement was to use the API of the Imgur\footnote{\url{https://imgur.com/}} platform.
Thanks to the official module that can be used with Python, the platform offers a simple way to upload the images to the service and returns the URL to the stored image when successfully uploaded.

    \subsection{Results terminating}\label{subsec:results-terminating}
When the model was executing the calculations multiple times, the results for each iteration were collected.
This is giving a really interesting insight into the model and its stability.
The tool called \mbox{\textit{ResultTerminator}} was serving this purpose.
The helper methods were used to determine whether the file is existing already and if not creating it.
To avoid concurrent saving problems, the file is achieving the lock and releasing it when the content is successfully saved to CSV file.

    \subsection{Slack notifier}\label{subsec:slack-notifier}
Working with PLGrid infrastructure and running a job on the Prometheus computing cluster was very helpful in terms of delegating the great amount of computing load onto external resources.
However, this solution has its downsides.
One of the major issues with this is the job queueing mechanism, was that it is not deterministic when the job will execute.
Therefore it requires constant manual checking for results.
To avoid that somehow, the authors prepared a custom mechanism, that can monitor the status of the job and notify the Prometheus user through the Slack\footnote{\url{https://slack.com/} – a messaging app for teams} channel.
Three following message templates were prepared for the purpose of notification:
\begin{enumerate}
    \item Simple message --- Generic message type build as a JSON template with help of Builder pattern to have flexibility in customizing the message.
    \item Pending job message --- able to notify about: job title, reporter, commit hash, job registration date and time, header for app preview, info message taken from a commit.
    Build as a special case of Simple type message.
    \item Results message --- The most complex message with all the statistic results produced as a result of the model computation. Prepared as a JSON template with the help of the Builder design pattern.
\end{enumerate}

Thanks to this tool, the time consumed for waiting for the previous version of model results could be spent effectively on refinements and implementation of further versions.

    % How to ref to paragraph: https://tex.stackexchange.com/a/334756
\paragraph{Statistics calculation}\label{para:statistics-calculation}
The prepared model was intended to be running several times in order to measure the consistency and stability of the results.
To properly measure and visualize the model scores, a dedicated module for processing the results was prepared.
The two main core parts can be distinguished --- the plotting utilities and the statistics metrics calculation submodule.

The submodule for calculating metrics allows for producing the mean value of accuracy, model loss, percentage of \gls{far} and \gls{frr}, number of true negatives, true positives, false negatives and false positives.
The plotting submodule allows for producing the following charts:
\begin{samepage}
    \begin{itemize}
        \item[---] generic linear plot creation used to display accuracy and loss charts,
        \item[---] creating the accuracy percentile histogram.
    \end{itemize}
\end{samepage}

The above metrics and functionalities are designed to produce the averaged model results for multiple executions.
The developed model is stable and the error is irrelevant between each execution.

    \section{Machine learning model}\label{sec:machine-learning-model}
The machine learning model prepared and trained on the collected dataset was the part of the data evaluation final part.
The dataset collected during the thesis preparation is not sufficient for a proper and full machine learning process.
Google Cloud Services are able to handle the deployment of the whole system efficiently.
However, this form of hosting and maintaining the infrastructure of the application comes with its price.
The whole infrastructure was running for about two months and during this time generated the cost of a total \$300.
This amount of cash is the highest financial outlay that could have been incurred by the authors of the work since no other scholarship than PLGrid computation cluster was granted for the purpose of preparing the thesis.
Longer exposure on the Web would cost extra money, that could not be afforded.
The other matter is that the engineering thesis defending has its deadline specified, thus the project has been carefully thought-out in the manner of time since the beginning of the implementation.
In order to meet the adopted milestones and goals, the data collection period had to follow strict deadlines.
The duration of the period when the data was collected could not be extended to broader terms.

The first remedy for a small amount of collected data was to use the transfer learning technique.
As stated in the "A survey of transfer learning" by Weiss et al. --- "In certain scenarios, obtaining training data that matches the feature space and predicted data distribution characteristics of the test data can be difficult and expensive.
Therefore, there is a need to create a high-performance learner for a target domain trained from a related source domain.
This is the motivation for transfer learning.
Transfer learning is used to improve a learner from one domain by transferring information from a related domain"~\cite{transfer-learning-def}.
The base feature vector for transferring to the bot recognition domain was taken from the TensorFlow Hub\upperref{itm:tensorflow-hub} website.
The model was pre-trained on ImageNet dataset\upperref{itm:image-net} using the architecture of Inception V3, which is showing great potential in the terms of the computer vision with improved performance over the previous versions, with a relatively modest computation cost~\cite{inception-v3}.
The model was built with the following dimensions of $299$\texttimes$299$\texttimes$3$ for the data input.
The topmost layers were added for the transferring and tuning to the bot detection domain.

The first attempts were not very exciting because of the limited amount of samples.
The model behaved unilaterally which means that the dominance of user's samples outshone the bot samples.
The different approaches were taken to improve the performance such as linear interpolation by connecting the points in recorded sequences.
Each used sequence originally consisted of many single discrete points without any additional pieces of information.
Interpolation provided an order between discrete coordinates and allowed feeding the neural network with additional information.

Another considered approach was a manipulation of the input data size.
The user's sequences were limited to the number of total bot sequences.
This solution was aimed to balance the dataset at the cost of fewer data.
The results of this approach turned out insufficient --- because of the total amount of bot sequences, the total size of the dataset drastically shrank, which resulted in a performance deterioration.
On the other hand, the duplication of the bot sequences was used.
The idea was similar to the one before, but instead of reducing, the number of bot samples was increased by using a single sample several times.
It resulted in an artificially balanced dataset.
However, this approach did not increase performance at all.

Manipulation of the distribution of labels between training and testing dataset was also considered.
It was done by performing either an equalization of the total number of both types in the testing dataset and the distribution of samples between both datasets.
The first solution did not affect model performance, but the second one slightly improved overall performance if the ratio was close to 50:50.
When the number of training samples was significantly greater than testing ones, the accuracy decreased due to a very small number of bot samples in testing dataset.

Yet another attempt to reduce the impact of the inappropriate dataset was changing the dataset itself.
The developed serializing tool made it possible to create a few datasets from recorded data with different minimal sequence length limits.
Using longer sequences meant that the overall number of them would be smaller.
The authors tested several ones and found out that the best performance was for a length equal to 50.

To overcome the described issue, another technique called data augmentation was used.
As stated in the "The Effectiveness of Data Augmentation in Image Classification using Deep Learning" by Perez et al. article --- "It is common knowledge that the more data an ML algorithm has access to, the more effective it can be.
Even when the data is of lower quality, algorithms can actually perform better, as long as useful data can be extracted by the model from the original data set"~\cite{augmentation}.
The main idea of this approach is to increase the samples of the dataset by using the same sample several times, but each time slightly transformed it.
The transformation is based on operations such as rotations, zooming, and flipping the original sample with some probability to do so, so each artificially generated sample should be slightly different than the others.
This approach allows to significantly increase the size of the dataset.
In the presented solution, it was done by development of augmentation script using Augmentor\upperref{itm:augmentor} package for Python.
Each of the datasets (users and bot) was extended to 30,000 samples which results in a total of 60,000 samples with just 639.
The ratio between the training and testing part was set to 80:20, so the network used for the training 48,000 samples and for the testing 12,000.
For the optimization of the model, the optimizer called Adam was used along with the learning rate exponential decay set initially to \num{1e-4}.
The decay rate was equal to 0.96 and the decay steps were set to \num{1e5}.
The loss function was set to the binary cross entropy loss due to the binary classification problem.

To save time and optimize resources usage, the first approach for running the model was to execute it in a multi-threaded way.
For this purpose, the module which was facilitating that was created.
Soon this approach was abandoned due to the encountered problems.
The main problem was that the SLURM manager is not able to split the resources in a way to the multithreaded approach to be sensible.
The resources were assigned to the whole computing node on which only 2 \gls{gpu} cards were able to be allocated.
This caused the problem of many machine learning models trying to run on a single card.

When run on the Prometheus cluster, the better approach was to split the computation between two \gls{gpu}s and run the model sequentially.
For this purpose, the Mirrored Strategy distribution technique from the TensorFlow library was used whereas training is split across multiple replicas.
Thanks to the split and the computing power of \gls{gpu}, single execution is faster, thus can be multiplied several times.
    \subsection{Model execution strategy}\label{subsec:model-execution-strategy}
As stated before, the model is running several times.
To save time and optimize resources usage, the first approach for running the model was to execute it in a multi-threaded way.
For this purpose, the module which was facilitating that was created.
Soon this approach was abandoned due to the encountered problems.
The main problem was that the SLURM manager is not able to split the resources in a way to the multithreaded approach to be sensible.
The resources were assigned to the whole computing node on which only 2 \gls{gpu} cards were able to be allocated.
This caused the problem of many machine learning models trying to run on a single card.

When run on the Prometheus cluster, the better approach was to split the computation between two \gls{gpu}s and run the model sequentially.
For this purpose, the Mirrored Strategy distribution technique from the TensorFlow library was used whereas training is split across multiple replicas.
Thanks to the split and the computing power of \gls{gpu}, single execution is fast, thus can be multiplied several times.

\chapter{Summary}\label{ch:summary}

\section{Summary}\label{sec:summary}
During the work on the presented solution, the authors took steps to limit the impact of the imbalanced dataset.
As an example, linear interpolation was used.
It was done by connecting points in recorded sequences.
Each single recorded sequence consisted of many single discrete points.
Interpolation provided an order between discrete coordinates and allowed feeding the neural network with additional information.

Another approach that was taken was a manipulation of the input data size.
On the one hand, the user's sequences were limited to the number of total bot sequences.
This solution was aimed to balance the dataset at the cost of fewer data.
The results of this approach turned out insufficient.
Because of the total amount of bot sequences, the total size of the dataset drastically shrank, which resulted in a worsening of performance.
On the other hand, the duplication of the bot sequences was used.
The idea was similar to the one before, but instead of reducing, the authors increased the number of bot samples using a single sample several times.
It resulted in an artificially balanced dataset.
However, this approach did not increase performance at all.

Manipulation of the distribution of labels between training and testing dataset was also considered.
It was done by equalization of the number of both types of samples in testing dataset as well as manipulating the distribution of the number of samples between both datasets.
The first solution did not affect, but the second one slightly improves overall performance if the ratio was close to \num{50}:\num{50}.
When the number of training samples was significantly greater than testing ones, the accuracy decreased due to a very small number of bot samples in testing dataset.

Yet another attempt to reduce the impact of the inappropriate dataset was changing the dataset itself.
The developed serializing tool made it possible to create a few datasets from recorded data with different minimal sequence length limits.
Using longer sequences meant that the overall number of them would be smaller.
The authors tested several ones and found out that the best performance was for a length equal to \num{50}, as it was mentioned before.

All of the presented approaches tended to minimize the dataset problem.
Some of them slightly improved performance and those were considered in the final solution.
Despite the efforts of the authors, the described problem significantly worsened the performance of the model.

\section{Further study}\label{sec:further-study}
Due to the described issues, the authors find further study mainly in the improvement of the dataset.
Some efforts may be taken to extend the size of the recorded data.
Firstly, recording sequences of a bigger group of users may be proposed as a solution, especially bot users to prevent imbalance.
Such a solution should improve the overall performance of the model or at least suggest other problems related to the quality of the dataset.
If the quality of the recorded samples would be inappropriate, the collection module~\footnote{\url{https://github.com/Mouse-BB-Team/Data-Collection}} should be reviewed and improved.
The main object of interest should be the method of gathering the samples.
Delays and synchronization that can disturb the reliability of the data may be considered as a major area of study.

Another approach that may be taken into consideration is another machine learning model.
In the presented solution, the focus was on the two-dimensional convolutional neural network, taking an example from related works like Chong et al.~\cite{Main} and Wei et al.~\cite{Inspiration}.
The work~\cite{Main} also shows other solutions, such as especially recursive neural networks.
Those kinds of neural networks are very popular in problems where the order and time intervals between samples have natural interpretations.
The problem which is described in this work also has similar properties.

These two described areas of study are found by the authors as the major to improve performance and reliability.
To deliver a safe and reliable solution to the commercial market further study is necessary.

% itd.
\appendix
\begin{appendices}
    \addcontentsline{toc}{section}{Appendix: Links}
    \section{Links}\label{sec:links}
\begin{enumerate}
    \item Balabit dataset - \url{https://github.com/balabit/Mouse-Dynamics-Challenge}\label{itm:balabit}
    \item Spring - \url{https://spring.io/}\label{itm:spring}
    \item Spring Boot - \url{https://spring.io/projects/spring-boot}\label{itm:spring-boot}
    \item Java Persistence API - \url{https://www.oracle.com/java/technologies/persistence-jsp.html}\label{itm:jpa}
    \item OAuth2 - \url{https://oauth.net/2/}\label{itm:oauth2}
    \item JSON Web Token - \url{https://jwt.io/}\label{itm:jwt}
    \item Colorlib - \url{https://colorlib.com/}\label{itm:colorlib}
    \item Creative Commons Attribution 3.0 License - \url{https://creativecommons.org/licenses/by/3.0/}\label{itm:license}
    \item Node.js - \url{https://nodejs.org/}\label{itm:node}
    \item Express framework - \url{https://expressjs.com/}\label{itm:express}
    \item Heroku Cloud Application Platform - \url{https://www.heroku.com/}\label{itm:heroku}
    \item Docker - \url{https://www.docker.com/}\label{itm:docker}
    \item Google Cloud Platform - \url{https://cloud.google.com/}\label{itm:gcp}
    \item Kubernetes - \url{https://kubernetes.io/}\label{itm:kubernetes}
    \item PLGrid - \url{http://www.plgrid.pl/}\label{itm:plgrid}
    \item Go Lang - \url{https://golang.org/}\label{itm:golang}
    \item Apache Avro - \url{https://avro.apache.org/}\label{itm:avro}
    \item Protocol Buffers - \url{https://developers.google.com/protocol-buffers}\label{itm:protobuf}
    \item Pandas - \url{https://pandas.pydata.org/}\label{itm:pandas}
    \item SLURM - \url{https://slurm.schedmd.com/documentation.html}\label{itm:slurm}
    \item Imgur platform - \url{https://imgur.com/}\label{itm:imgur}
    \item Slack - \url{https://slack.com/}\label{itm:slack}
    \item Tensorflow Hub - \url{https://www.tensorflow.org/hub}\label{itm:tensorflow-hub}
    \item ImageNet dataset - \url{http://www.image-net.org/}\label{itm:image-net}
    \item Data Collection submodule - \url{https://github.com/Mouse-BB-Team/Data-Collection}\label{itm:data-collection}
    \item Bot Detection submodule - \url{https://github.com/Mouse-BB-Team/Bot-Detection}\label{itm:bot-detection}
\end{enumerate}

\end{appendices}


% \include{dodatekB}
% itd.
%\clearpage
\printglossary
\printbibliography[heading=bibintoc]
\thispagestyle{bibliography}


\end{document}
