\documentclass[en,11pt]{classes/aghdpl}  % praca w języku angielskim

% Lista wszystkich języków stanowiących języki pozycji bibliograficznych użytych w pracy.
% (Zgodnie z zasadami tworzenia bibliografii każda pozycja powinna zostać utworzona zgodnie z zasadami języka, w którym dana publikacja została napisana.)
\usepackage[english]{babel}

\usepackage[utf8]{inputenc}

% dodatkowe pakiety

\usepackage{mathtools}
\usepackage{amsfonts}
\usepackage{amsmath}
\usepackage{amsthm}
\usepackage{hyperref}
\usepackage{xcolor}
\usepackage{listings}
\usepackage{minted}


% --- < bibliografia > ---

\usepackage[
style=numeric,
sorting=none,
%
% Zastosuj styl wpisu bibliograficznego właściwy językowi publikacji.
language=autobib,
autolang=other,
% Zapisuj datę dostępu do strony WWW w formacie RRRR-MM-DD.
urldate=iso8601,
% Nie dodawaj numerów stron, na których występuje cytowanie.
backref=false,
% Podawaj ISBN.
isbn=true,
% Nie podawaj URL-i, o ile nie jest to konieczne.
url=false,
%
% Ustawienia związane z polskimi normami dla bibliografii.
maxbibnames=3,
% Jeżeli używamy BibTeXa:
backend=bibtex
]{biblatex}

\usepackage{csquotes}
% Ponieważ `csquotes` nie posiada polskiego stylu, można skorzystać z mocno zbliżonego stylu chorwackiego.
\DeclareQuoteAlias{croatian}{english}

\addbibresource{bibliography/bibliography.bib}

% Nie wyświetlaj wybranych pól.
%\AtEveryBibitem{\clearfield{note}}


% ------------------------
% --- < listingi > ---

% --- < json listing > ---
\definecolor{delim}{RGB}{20,105,176}
\definecolor{numb}{RGB}{106, 109, 32}
\definecolor{string}{rgb}{0.64,0.08,0.08}
\definecolor{bg}{rgb}{246, 246, 246}

\lstdefinelanguage{json}{
    numbers=left,
    numberstyle=\small,
    frame=single,
    rulecolor=\color{black},
    showspaces=false,
    showtabs=false,
    breaklines=true,
    captionpos=b,
    postbreak=\raisebox{0ex}[0ex][0ex]{\ensuremath{\color{gray}\hookrightarrow\space}},
    breakatwhitespace=true,
    basicstyle=\ttfamily\small,
    upquote=true,
    morestring=[b]",
    stringstyle=\color{string},
    literate=
     *{0}{{{\color{numb}0}}}{1}
      {1}{{{\color{numb}1}}}{1}
      {2}{{{\color{numb}2}}}{1}
      {3}{{{\color{numb}3}}}{1}
      {4}{{{\color{numb}4}}}{1}
      {5}{{{\color{numb}5}}}{1}
      {6}{{{\color{numb}6}}}{1}
      {7}{{{\color{numb}7}}}{1}
      {8}{{{\color{numb}8}}}{1}
      {9}{{{\color{numb}9}}}{1}
      {\{}{{{\color{delim}{\{}}}}{1}
      {\}}{{{\color{delim}{\}}}}}{1}
      {[}{{{\color{delim}{[}}}}{1}
      {]}{{{\color{delim}{]}}}}{1},
}

% Użyj czcionki kroju Courier.
\usepackage{courier}

\usepackage{listings}
\lstloadlanguages{TeX}

\lstset{
	literate={ą}{{\k{a}}}1
           {ć}{{\'c}}1
           {ę}{{\k{e}}}1
           {ó}{{\'o}}1
           {ń}{{\'n}}1
           {ł}{{\l{}}}1
           {ś}{{\'s}}1
           {ź}{{\'z}}1
           {ż}{{\.z}}1
           {Ą}{{\k{A}}}1
           {Ć}{{\'C}}1
           {Ę}{{\k{E}}}1
           {Ó}{{\'O}}1
           {Ń}{{\'N}}1
           {Ł}{{\L{}}}1
           {Ś}{{\'S}}1
           {Ź}{{\'Z}}1
           {Ż}{{\.Z}}1,
	basicstyle=\footnotesize\ttfamily,
}

% ------------------------

\AtBeginDocument{
	\renewcommand{\tablename}{Tab.}
	\renewcommand{\figurename}{Fig.}
}

% ------------------------
% --- < tabele > ---

\usepackage{array}
\usepackage{tabularx}
\usepackage{multirow}
\usepackage{booktabs}
\usepackage{makecell}
\usepackage[flushleft]{threeparttable}

% defines the X column to use m (\parbox[c]) instead of p (`parbox[t]`)
\newcolumntype{C}[1]{>{\hsize=#1\hsize\centering\arraybackslash}X}


%---------------------------------------------------------------------------

\author{Kamil~Kaliś and Piotr~Kuglin}
\shortauthor{K.~Kaliś P.~Kuglin}

% \titlePL{Przygotowanie bardzo długiej i pasjonującej pracy dyplomowej w~systemie~\LaTeX}
% \titleEN{Protection of Web Applications with Behavioral Biometrics}

\titlePL{}
\titleEN{Protection of Web Applications with Behavioral Biometrics}


% \shorttitlePL{Przygotowanie pracy dyplomowej w~systemie \LaTeX} % skrócona wersja tytułu jeśli jest bardzo długi
\shorttitleEN{Protection of Web Applications with Behavioral Biometrics}

% \thesistype{Praca dyplomowa magisterska}
\thesistype{Bachelor's Thesis}

% \supervisor{prof. dr hab. Marcin Szpyrka}
\supervisor{Piotr Chołda, PhD}

% \degreeprogramme{Informatyka}
\degreeprogramme{Electronics and Telecommunications}

\date{2020}

% \department{Katedra Informatyki Stosowanej}
\department{Department of Telecommunications}

% \faculty{Wydział Elektrotechniki, Automatyki,\protect\\[-1mm] Informatyki i Inżynierii Biomedycznej}
\faculty{Faculty of Computer Science, Electronics and Telecommunications}

\acknowledgements{We hereby express our thanks to our supervisor – Piotr Chołda, PhD, who was guiding the process of creation of this thesis. The support in granting the needed resources for the study and feedback provided by Mr. Chołda are invaluable.\\
Special thanks to Krzysztof Rusek, PhD – for consulting the approach for proper machine learning model execution.\\
This research was supported in part by PLGrid Infrastructure.
}


\setlength{\cftsecnumwidth}{10mm}

%---------------------------------------------------------------------------
\setcounter{secnumdepth}{4}
\brokenpenalty=10000\relax

\begin{document}

\titlepages

% Ponowne zdefiniowanie stylu `plain`, aby usunąć numer strony z pierwszej strony spisu treści i poszczególnych rozdziałów.
\fancypagestyle{plain}
{
	% Usuń nagłówek i stopkę
	\fancyhf{}
	% Usuń linie.
	\renewcommand{\headrulewidth}{0pt}
	\renewcommand{\footrulewidth}{0pt}
}

\setcounter{tocdepth}{2}
\tableofcontents
\clearpage

\chapter{Implementation}\label{ch:implementation}
To fulfill assumptions of this work, the authors prepare a whole system which is able to collect and verify the mouse dynamics data.
The high-level flow of the system is presented in Fig.~\ref{fig:overall_system_structure} and provides the basis for further considerations.
\begin{figure}[!hbt]
    \includegraphics[width=\linewidth]{resources/overall_system_structure}
    \captionof{figure}{Overall system structure diagram}
    \label{fig:overall_system_structure}
\end{figure}

The entry point for the system is user's browser that allows the human user as well as human impersonating bot for access to the prepared website which is served in the Internet.
The \mbox{Data Collection}\upperref{itm:data-collection} module which is persisted and operates within a cloud acts as a mouse dynamics data collector, which means that every single mouse event generated by user on the website is intercepted, transformed and stored in the underlying database (Fig.~\ref{fig:overall_system_structure}, pt. 1 and pt.2).
The administrator of presented system is able to retrieve the data from the database in any time (pt. 3).
The downloaded data can be further processed (pt. 4) to the dataset which will be used in machine learning stage.
The administrator of the system should prepare a machine learning model and upload it to the Git repository (pt. 5) which will be then obtained by computing cluster to perform computation using the model from the observed branch.
Such an approach allows to work on the solution simultaneously by many data scientists and makes it possible to accelerate the research.
The dataset should be uploaded to the computing cluster and persisted in the group's storage that allows to use it by many different paralleled computations (not included in the diagram due to decrease of readability).
Each computation is requested from the local computer using prepared Git's alias that performs sequence of operations such us establishing the connection to the cluster, fetching the current version of code from the repository and submitting the job to the Slurm\upperref{itm:slurm} workload manager (pt. 6, 7, 8).
The work of the cluster is fully asynchronous because each job is queued and therefore the completion time is unknown which is very inconvenient from the authors' point of view because according to the authors' knowledge there not exists any notification system which allows getting the information about the finished job.
To overcome these limitations, the notification system is proposed as a part of Bot Detection\upperref{itm:bot-detection} module.
The creation of such system enables the possibility for presenting the results of the computed job in the notification message as well as the graphs prepared based on the output of the job.
To provide the possibility of sending the images, the notification system uses the external image hosting website called Imgur\upperref{itm:imgur} which allows to upload pictures to the server and host them under the generated \gls{url} (pt. 11).
The results and the graphs are combined into Slack's\upperref{itm:slack} message and then sent to the previously prepared Slack's channel by using the Slack's webhook \gls{api} (pt. 12, 13).

The further subchapters treats about the implementation details, at the beginning the Data Collection module is described along with the cloud configuration and performance tests, further, the bot which impersonates the human user and finally the machine learning model alongside the tools such as, among others, a notification module.


\section{Presentation and reverse proxy}\label{sec:reverse-proxy}
The second core submodule created for the purpose of collecting the data from the users is the frontend application exposed the internet. This can be split further into two different functional modules~--- client and server side.

\subsection{Client side module}\label{subsec:client-side-module}
The client-side module is an end-user presentation layer that is built with the HTML5\footnote{\url{https://html5.org/}}~and~CSS3\footnote{\url{https://www.w3.org/Style/CSS/Overview.en.html}}~--- technologies that are the core and standard for modern web applications nowadays.
HTML is a widely used markup language used to create hypertext documents and CSS is used as a style-sheet for adding the design for web documents.
The content of the website is a free, prebuilt template taken from the Colorlib\footnote{\url{https://colorlib.com/}} collection.
Offered web templates are licensed under the CC BY 3.0 License \footnote{Creative Commons Attribution 3.0 License --- \url{https://creativecommons.org/licenses/by/3.0/}}.
The mechanism that collects users' mouse actions is designed as a "plugin" script, meaning that the template can be changed easily, so different styled environments that serve various purposes can be used to collect the data.

To use the system and participate in the research, interested user is required to first accept the consents of usage of the website as well as accept the usage of the cookies.
After that, the register option is allowed and required.
Next, the user is redirected to the login page.
To persists consistency of the registration and login, the custom static web pages were created to be easily transferable between the templates.
On each document, the FAQ panel with more information regarding the project is exposed as a drop-down list.
To authenticate, users are required to provide the credentials to log in to the system.
These credentials are then sent through https --- which means that they are secured and encrypted --- to the reverse proxy server which then performs some action to authenticate a user, sets the cookie with granted JWT\footnote{\url{https://jwt.io/}} token and redirects the user to the homepage.
More on the user authentication and the token obtaining sequence is described in
The whole sequence is described and shown in subsection \ref{subsec:server-side-module}).

After a successful login, the token allows for site usage and ensures the identity of the user.
From now on, the actions performed by the user are recorded into the batches and sent to the API every two seconds.
The event listeners are awaiting four different action types: mouse move, mouse-down, mouse-up, mouse wheel action.
Collected actions are packed into JSON object and sent to the server side API (described in \ref{subsec:server-side-module}) --- the fields included are shown in the Listing~\ref{listing:mouse-events-json-schema}.

\lstinputlisting[language=json, caption=JSON schema for mouse event batches, label=listing:mouse-events-json-schema]{resources/mouse-event-json-schema.json}

\subsection{Server side module}\label{subsec:server-side-module}
The server side module is build with Node.js\footnote{\url{https://nodejs.org/}} runtime with the Express\footnote{\url{https://expressjs.com/}} framework on the top of it.
Node.js is an asynchronous JavaScript runtime, which is widely used to build high-end, scalable commercial applications.
Express is light-weight and simple to use, yet powerful web framework for Node.js, which allows for a fast and convenient HTTP server set-up server for serving the static web documents over the internet.
This module is serving the purpose of reverse proxy between clients and the backend API.

The main responsibilities of the reverse proxy are signing up the user, token granting, validation and caching, serving static web documents storing and passing the data to for persistence.

When a new user tries to sign-up, secured with TLS credentials from the sign-up form are sent to one of the proxy endpoints, where they are then passed to the backend API.
Because the backend is not exposed anywhere over the public network, there is no need to secure the credentials with TLS.\\
The user is granted JWT Access and Refresh tokens after logging in using the login form.
The user credentials are secured with TLS and received by proxy API, whereas the proxy server is additionally appending the Client id and Client secret for OAuth2 server to the request "Authorization" header, as the proxy server authenticates to OAuth2 server with HTTP Basic authentication scheme.
Credentials of the user that wishes to log in are included in the body of the request.
When the user is properly authenticated, the OAuth2 server responds with a valid JWT token which is then set as a cookie with HttpOnly, Secure and SameSite:~strict options.
When the token is successfully granted, it is also saved to the Redis, which is a very efficient key-value NoSQL database.
The retrieval of such a token is very fast, so this is serving the purpose of the caching system, which results in a great efficiency improvement and lower response time of the server.
For each interaction and request for resources, the user has to hold a valid JWT Token.
First, the token is searched in the cache database. If it does not figure there, the proxy server is attempting to check the token with the backend OAuth2 server.
If the access token is not set on the user request or the server responds with Bad Request status, the refresh token is being used for access token renewal.
The complete seqnece visualization can be seen in \mbox{Fig.~\ref{fig:jwt-sequence}}.
The proxy server is storing the user mouse data received from the client-side and periodically passes it to the backend API for persistence.




% Propably should be included elsewhere, maybe appendix?
\begin{figure}[!hbt]
    
    \centering
    \includegraphics[width=\linewidth]{resources/jwt_sequence_diagram.png}
    \captionsetup{width=\linewidth}
    \captionof{figure}{JWT token obtaining sequence}
    \label{fig:jwt-sequence}
\end{figure}
\chapter{Summary}\label{ch:summary}

\section{Summary}\label{sec:summary}
During the work on the presented solution, the authors took steps to limit the impact of the imbalanced dataset.
As an example, linear interpolation was used.
It was done by connecting points in recorded sequences.
Each single recorded sequence consisted of many single discrete points.
Interpolation provided an order between discrete coordinates and allowed feeding the neural network with additional information.

Another approach that was taken was a manipulation of the input data size.
On the one hand, the user's sequences were limited to the number of total bot sequences.
This solution was aimed to balance the dataset at the cost of fewer data.
The results of this approach turned out insufficient.
Because of the total amount of bot sequences, the total size of the dataset drastically shrank, which resulted in a worsening of performance.
On the other hand, the duplication of the bot sequences was used.
The idea was similar to the one before, but instead of reducing, the authors increased the number of bot samples using a single sample several times.
It resulted in an artificially balanced dataset.
However, this approach did not increase performance at all.

Manipulation of the distribution of labels between training and testing dataset was also considered.
It was done by equalization of the number of both types of samples in testing dataset as well as manipulating the distribution of the number of samples between both datasets.
The first solution did not affect, but the second one slightly improves overall performance if the ratio was close to \num{50}:\num{50}.
When the number of training samples was significantly greater than testing ones, the accuracy decreased due to a very small number of bot samples in testing dataset.

Yet another attempt to reduce the impact of the inappropriate dataset was changing the dataset itself.
The developed serializing tool made it possible to create a few datasets from recorded data with different minimal sequence length limits.
Using longer sequences meant that the overall number of them would be smaller.
The authors tested several ones and found out that the best performance was for a length equal to \num{50}, as it was mentioned before.

All of the presented approaches tended to minimize the dataset problem.
Some of them slightly improved performance and those were considered in the final solution.
Despite the efforts of the authors, the described problem significantly worsened the performance of the model.

\section{Further study}\label{sec:further-study}
Due to the described issues, the authors find further study mainly in the improvement of the dataset.
Some efforts may be taken to extend the size of the recorded data.
Firstly, recording sequences of a bigger group of users may be proposed as a solution, especially bot users to prevent imbalance.
Such a solution should improve the overall performance of the model or at least suggest other problems related to the quality of the dataset.
If the quality of the recorded samples would be inappropriate, the collection module~\footnote{\url{https://github.com/Mouse-BB-Team/Data-Collection}} should be reviewed and improved.
The main object of interest should be the method of gathering the samples.
Delays and synchronization that can disturb the reliability of the data may be considered as a major area of study.

Another approach that may be taken into consideration is another machine learning model.
In the presented solution, the focus was on the two-dimensional convolutional neural network, taking an example from related works like Chong et al.~\cite{Main} and Wei et al.~\cite{Inspiration}.
The work~\cite{Main} also shows other solutions, such as especially recursive neural networks.
Those kinds of neural networks are very popular in problems where the order and time intervals between samples have natural interpretations.
The problem which is described in this work also has similar properties.

These two described areas of study are found by the authors as the major to improve performance and reliability.
To deliver a safe and reliable solution to the commercial market further study is necessary.

% itd.
% \appendix
% \include{dodatekA}
% \include{dodatekB}
% itd.

\printbibliography

\end{document}
